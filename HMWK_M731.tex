\magnification=1100
\input amstex
\documentstyle{amsppt}
\pageheight{22truecm}
\pagewidth{15.6truecm}
\vcorrection{0.3truecm}
\hcorrection{0.4truecm}

\documentstyle{amsppt}
\NoBlackBoxes
%\NoPageNumbers

{\catcode`@=11
\gdef\nologo{\let\logo@\empty}
\catcode`@=12}
\nologo

\def \Box{\square}
\def\nin{\noindent}

\def \IT{\Bbb T}
\def \al {\alpha}
\def \De {\Delta}
\def\Mc{M\raise0.7ex\hbox{\eightrm c}}
\def\bu {\bullet}
\def\bi{ {\tfrac{1}{b}}}
\def\dbi{{\tfrac{1}{b}}}
\def\A{{\Cal A}}
\def\H{{\Cal H}}
\def\B{{\Cal B}}
\def\E{{\Cal E}}
\def\L {{\Cal L}}
\def\D{{\Cal D}}
\def\R{{\Cal R}}
\def \M {{\Cal M}}
\def\N{{\Cal N}}
\def \SS{{\Cal S}}
\def \T {{\Cal T}}
\def\V{{\Cal V}}
\def\X{{\Cal X}}
\def\Y{{\Cal Y}}
\def\W{{\Cal W}}
\def\K{{\Cal K}}
\def\O{{\Omega}}
\def\IS{{\Bbb S}}
\def \IC {{\Bbb C}}
\def \IP{{\Bbb P}}
\def\RR{{\Bbb R}}

\def \IR{{\Bbb R}}
\def\DD{\bold D}
\def\IZ{{\Bbb Z}}
\def\IH{{\Bbb H}}
\def\l{\lambda}
\def\z{\zeta}
\def\a{\alpha}
\def\g{\gamma}
\def\gA{{\frak A}}
\def\gB{{\frak B}}
\def\gF{{\frak F}}
\def\gG{{\frak G}}
\def\gH{{\frak H}}
\def\gN{{\frak N}}
\def\gP{{\frak P}}
\def\gT{{\frak T}}
\def \gg {{\frak g}}

\def\ddx{\tfrac d {dx}}
\def\ddy{\tfrac d {dy}}
\def\deriv#1{{\tfrac{d#1}{dx}}}
\def\mod#1{\left | #1\IRight |}

\def\fin{\qquad \blacksquare }

\def\wave{\square}

\def\ch{\raise 0.3ex\hbox{$\chi$}\kern-.15em}

\def\norm#1{\left\Vert{\, #1 \, }\right \Vert}

\def\card{\operatorname{card}}
\def\intinfin{\int_{-\infty}^{\infty}}
\def\const{\operatorname{const.}}
\def\dist{\operatorname{dist}}
\def\con{\operatorname{const}}
\def\sgn{\operatorname{sgn}}
\def\supp{\operatorname{supp}}
\def\sign{{\rm sgn}}

%open and closed sectors and more
\def\osect#1{S_{\kern-.05em #1}^0}
\def\csect#1{S_{\kern-.05em #1}}
\def\Hinf{H^\infty}
\def\Hinfos#1{H^\infty(\osect#1)}
\def\Psios#1{\Psi(\osect#1)}
\def\inv{^{-1}}
\def\sgn{\text{sgn}}
\def\half{{\ssize{\frac12}}}


%\hcorrection{.3in}
\vskip .2in
\centerline{ \bf M731--Partial Differential Equations \,   HOMEWORKS -- Fall 2019}
\vskip .1in
\centerline{ Prof. Andrea R. Nahmod}
\document
\vskip .2in


\centerline{ {\bf{ SET 1: Due  Thursday 10/03/2019} }}
\vskip .2in 

{\bf From McOwen's Book: read the introduction \& Sections 1.1d) and 1.3}

\vskip .2in 
 

{\bf From McOwen's Book Section 1.1 do:\quad   1, 2, 3, 4a)b), 6a) 9.} 

\vskip .2in 

{\bf Additional Problems.}  
\vskip .1in 


 (1) \, Write down an explicit formula for a function $u$ solving the inhomogeneous initial value problem 
$$\cases  u_t + b \cdot \nabla u = \, f \qquad \text{on} \quad \IR^n \times (0,\infty) \\
\qquad \qquad  u =\, g \qquad \text{on} \quad \IR^n \times \{t=0\}   \endcases $$
where $f = f(x,t)$, $f: \IR^n \times [0, \infty) \to \IR$, $c \in \IR$, $b \in \IR^n$ and $g : \IR^n \to \IR$ are all given.
(Hint.  Solve first the homogeneous problem (i.e. $f =0$) using the method of characteristics (as in class or  section 1.1a) ).  Then use the Fundamental Theorem of Calculus to write $u(x,t)-g(x-bt)$ for $(x,t)$ along a characteristic  ($g(x-bt)$ is the homogeneous solution) as an appropriate integral of $f$.  At the end $u$ will be the sum of the homogeneous solution plus an integral term.)   

 \vskip .2in 
 
 
 (2) \, Write down an explicit formula for a function $u$ solving the initial value problem 
$$\cases u_t + b \cdot \nabla u + c \, u =\, 0 \qquad \text{on} \quad \IR^n \times (0,\infty) \\
\qquad \qquad \quad  \quad u =\, g \qquad \text{on} \quad \IR^n \times \{t=0\} \endcases $$
where $c \in \IR$, $b \in \IR^n$ and $g : \IR^n \to \IR$ are given.

\vskip .2in 

 (3) \, Let $f$ be a continuous function on an open set $D \subset \IR^n$ such that 
 $$ \int_{D_0}  \, f(x) \, dx \, =\, 0 \qquad \quad \text{ for all }\quad  D_0 \subset D.$$
 
Prove that then $f \equiv 0$ on $D$.  

\vskip .3in

\centerline{ {\bf{ SET 2. Due:  Thursday 10/17/2019} }}

\vskip .2in



{\bf From McOwen's Book Section 2.1:\quad  1, 2, 7 } 

\vskip .2in

{\bf From McOwen's Book Section 2.2:} \, Read examples in 2.2a); read all of Section 2.2b). Then  {\bf do \, $1$}.


\vskip .2in

{\bf Additional Problems.}  
\vskip .05in 


1) a) Find the characteristics of the PDE 
$$ y^2 u_{xx}  - 2 y u_{xy}  +    u_{yy} =  u_x + 6y, $$ and determine if elliptic, parabolic or hyperbolic.

b) Then find the canonical form and use it to find the solution $u$ first in the $\xi$ and $\eta$ variables and then 
in the $x$ and $y$ variables. 

\vskip .2in 

2) a) Find the characteristics of the PDE 
$$ x u_{xx} + (x-y) u_{xy} - y  u_{yy} = 0, \qquad x>0, y>0, $$ and determine if elliptic, parabolic or hyperbolic.

b) Then show that it can be transformed into the canonical form 
$$ (\xi^2 + 4 \eta) u_{\xi\eta} + \xi u_{\eta} = 0 $$
for $\xi$ and $\eta$ are suitably chosen canonical coordinates. 
and use this to obtain the general solution 
in the $\xi$ and $\eta$ variables. 

\vskip .2in 

{\bf Robert McOwen's Book Section 3.1:   \, 1, 4, 5}  

\vskip .1in

{\bf Additional Problems.} 
\vskip .1in 
(1)  (a)Show that the general solution to the PDE $u_{xy}=0$  is
$$ u(x,y) = F(x) + G(y)$$ 

for arbitrary functions $F,\, G$
\vskip .1in
(b) Using a change of variables $\xi = x+t$ and $\eta = x -t$, show that $$ u_{tt} - u_{xx} =0 \quad \text{ if and only if } \quad u_{\xi \eta} = 0$$
\vskip .1in
(c) Use parts (a) and (b) to rederive D'Alembert's formula.

\vskip .2in 

(2) Let $u \in C^2(\IR \times [0, \infty))$ be a solution to the Cauchy initial value problem  $$\cases u_{tt} - u_{xx} = 0  \qquad \text{in} \quad \IR \times (0, \infty)\\
u(0, x) = g(x)  \qquad \text{in} \quad \IR \\
u_t(0, x) = h(x)  \qquad \text{in} \quad  \IR  \endcases $$ Suppose that $g$ and $h$ are smooth and have compact support.  The {\it kinetic energy} is $$k(t):= \frac{1}{2} \int_{\IR} \, u_t^2(t, x) \, dx $$ and the {\it potential energy } is 
$$p(t):= \frac{1}{2} \int_{\IR} \, u_x^2(t, x) \, dx $$ 
Prove : 

(a) k(t) + p(t) is constant in $t$.
\vskip .1in 
(b)  k(t) = p(t) for all {\it large enough} times $t$.
 
\vskip .2in


\centerline{ {\bf{ SET 3 Due:  Tuesday 10/29/2019}}}
\vskip .2in 



{\bf Read first Robert McOwen's Book Section 2.3. Then do problems:   \, 4, 8, 10, 11c).}


{\bf Additional Problems on Distributions.} 
\vskip .2in

(1).  Prove the Remark at the bottom of page 10 of the Notes on Distributions.  This is a generalization of Pb 8 in section 2.3.
\vskip .2in

A generalized version of this statement is the following problem: (how would you do it?)

 \vskip .2in

(1') Let $\{ f_j\} \in L^1(\IR^n)$ be a sequence of  nonnegative functions such that 
$$\align &\int_{\IR^n} \, f_j(x) \, dx \to 1 \quad \text{ as} \quad   j \to \infty \quad \text{ and for any }  a>0,  \\ 
&\int_{|x| > a} \, f_j(x) \, dx \to 0 \quad \text{ as} \quad j \to \infty . \endalign$$Let $F_{f_j}$ be the distribution defined by $f_j$.  Prove that $F_{f_j} \to \delta$ in ${\Cal D}^{\prime}$. 

\vskip .2in

(2)  For $\delta$ the Dirac delta distribution, find the $k$-th (weak) derivative $\delta^{(k)}$ for any $k \geq 1$. 

\vskip .2in

(3)  Find the derivative in the sense of distributions of the $\text{sgn} x$, the function on $\Bbb R$ defined to be
equal to $1$ for $x >0$, equal to $0$ for $x=0$ and equal to $-1$ for $ x<0$.

\vskip .2in

(4)  Compute $\frac{d}{dx} ( \log |x|) $  on $\IR$  in the sense of distributions. 
\vskip .05in

\noindent Recall that the principal value of $\frac{1}{x}$   ( pv  $\frac{1}{x}$ )  is defined as 
$$ \langle  \text{pv} \frac{1}{x} , \phi \rangle = \text{pv} \int_{\IR} \frac{ \phi(x)}{x} \, dx := \lim_{\varepsilon \to 0^+} \int_{ |x| > \varepsilon} \,  \frac{ \phi(x)}{x} \, dx $$


\vskip .2in

(5)  Prove that the Dirac delta distribution  $\delta_{x_0}$  with point mass at $x_0 \in \IR^n$ (fixed) is not given by a locally integrable function.
In other words prove that there does not exist any $f \in L^1_{\text loc} (\IR^n)$ such that 
$$ \langle \delta_{x_0}, v \rangle = \langle f , v \rangle \qquad \text{ for all } \quad v \in C^{\infty}_0(\IR^n) $$

\vskip .1in

\vskip .3in


%\centerline{ {\bf{ SET 4} }}
%\vskip .1in
%{\bf Robert McOwen's Book Section 3.1:   \, 5 }
%
%\centerline{ {\bf{ SET 3} }}
%
%\vskip .1in
%
%{\bf Robert McOwen's Book Section 3.1:   \, 1, 4}  
%
%\vskip .1in
%
%{\bf Additional Problems.} 
%\vskip .1in 
%(1)  (a)Show that the general solution to the PDE $u_{xy}=0$  is
%$$ u(x,y) = F(x) + G(y)$$ 
%
%for arbitrary functions $F,\, G$
%\vskip .1in
%(b) Using a change of variables $\xi = x+t$ and $\eta = x -t$, show that $$ u_{tt} - u_{xx} =0 \quad \text{ if and only if } \quad u_{\xi \eta} = 0$$
%\vskip .1in
%(c) Use parts (a) and (b) to rederive D'Alembert's formula.
%
%\vskip .2in 
%
%(2) Let $u \in C^2(\IR \times [0, \infty))$ be a solution to the Cauchy initial value problem  $$\cases u_{tt} - u_{xx} = 0  \qquad \text{in} \quad \IR \times (0, \infty)\\
%u(0, x) = g(x)  \qquad \text{in} \quad \IR \\
%u_t(0, x) = h(x)  \qquad \text{in} \quad  \IR  \endcases $$ Suppose that $g$ and $h$ are smooth and have compact support.  The {\it kinetic energy} is $$k(t):= \frac{1}{2} \int_{\IR} \, u_t^2(t, x) \, dx $$ and the {\it potential energy } is 
%$$p(t):= \frac{1}{2} \int_{\IR} \, u_x^2(t, x) \, dx $$ 
%Prove : 
%
%(a) k(t) + p(t) is constant in $t$.
%\vskip .1in 
%(b)  k(t) = p(t) for all {\it large enough} times $t$.
% 
\vskip .2in

%\enddocument
\centerline{ {\bf{ SET 4} Due: Thursday  11/07/2019}}
\vskip .1in


{\bf Robert McOwen's Book Section 3.2:   \, 1, 2, 2*, 3, 5, 6a)}

\vskip .1in 
{\bf Hint for Problem 2} First show that if $y=(y_1, y_2, y_3)$  is a point in the unit sphere $\IS^2$ then $$ \int_{\IS^2} \, y_j \, d\sigma(y) \, =\, 0 \qquad  \quad j=1, 2, 3$$ where as always $d \sigma$ is the area surface element. This can be proved by an explicit calculation using spherical coordinates or also by symmetry, splitting for each $j$, the integral over the sphere $\IS^2$ into the two integrals for the half spheres $y_j \geq 0$ and $y_j \leq 0$ and showing the two integrals cancel out).

\vskip .1in 

{\bf Problem 2*} Same as problem 2 but with initial conditions $u(x,0)=0$ and $u_t(x,0) = x_2$. 
\vskip .2in 
{\bf Hint for Problem 5} Follow the hint in the back of the book and define $u(x_1, x_2, x_3) = \cos(\frac{m}{c} x_3) v (x_1, x_2, t)$. Then prove by a direct calculation $u$ satisfies the $3d$ (linear homogeneous) wave equation. Hence $u$ can be represented in terms of $g$ and $h$ using Kirchhoff's formula. Assume first for simplicity that $g=0$ and write the explicit formula for $u$ in this case. Then set $x_3=0$ and proceed as in the `method of descent' (parametrize the two halves of  $\IS^2$ corresponding by graphs   $y_3= \pm \sqrt{1 -(y_1^2 + y_2^2)}$) to 
obtain a formula for $v$ which should look like: 
$$ v (x_1, x_2, t) = C \, t \, \int_D \, \frac{ \cos\biggl( m t \sqrt{ 1 -(y_1^2 + y_2^2)}\biggr) \, h(x_1 + c t y_1, x_2 + ct y_2) }{  \sqrt{ 1 -(y_1^2 + y_2^2)} } \, dy_1 dy_2 $$ where $D:=\{ (y_1, y_2) : \, y_1^2 + y_2^2 \leq 1\, \}$.

Finally drop the assumption that $g=0$ to get the general formula. 

\vskip .2in 

{\bf Hint for Problem 6a)} The decay in $t$ in $3d$ is due to waves spreading out in space on expanding spheres $\partial B(x, ct)$ as $t \to \infty$. This bound reflects the {\it dispersion} of waves.  Assume first $g=0$ and use Kirchhoff's formula in conjunction with the fact that $h$ is bounded (i.e. \,  $|h(x)| \leq M$ for some $M>0$) and compactly supported ( hence $h(x) =0$ if $|x| \geq R$ for some large $R>0$). From these (plus change of variables) one can prove that 
$$| u(x, t)| \leq \frac{M}{ C t} \text{area}(\partial B(x, ct) \cap B(0, R))$$
where $C>0$ depends on the area of the unit sphere and the speed $c$. Next show that the area of the {\it spherical cap} $\partial B(x, ct) \cap B(0, R)$ can be bounded by some absolute constant times $R^2$- and hence by a quantity that is independent of $x$ and $t$.  Next note that if $h=0$ instead then both terms in $g$ in Kirchhoff's formula can be treated similarly as the previous case (do it!). Finally, for $g, h \in C^{\infty}_0(\IR^3)$ we have the sum of 3 terms all of which we know how to treat. 


\vskip .1in 
{\bf Note related to part 6b)}. In $2d$ there is `one less direction' than in $3d$ for waves to spread out so intuitively we expect the amplitude of the waves to decay slower as time increases.  And indeed, in $2d$ there is decay bound of the form $| u(x, t)| \leq \frac{C}{\sqrt{t}}$ but this is harder to prove.  In $1d$ , however, as we can clearly see from D'Alembert's formula there is no decay at all as $t \to \infty$. 

\vskip .2in 

%\centerline{ {\bf{ SET 5. Due TBA} }}
%\vskip .1in
%

{\bf Robert McOwen's Book Section 3.3:   \, 1, 2, 4, 5. }
\vskip .1in 
{\bf Additional Problem.} Let $f: \IR^n \to \IR$ be a continuous function and let $x\in \IR^n$ be fixed. For $r >0$ let 
$$  B_r(x):=\, \{ \, y \in \IR^n \, :\, | x-y | \, \leq \, r\, \}, \quad\text{and}\quad \partial B_r(x):= \{ \, y \in \IR^n \, :\, | x-y | \, =\, r\, \}.$$ {\bf a)} Prove that $$ \frac{d}{dr} \int_{B_r(x)} \, f(y) \, dy \,=\, \int_{\partial B_r(x)} \, f(y) \, d\sigma(y) $$ where $d\sigma$ is the surface measure. 

\noindent {\bf Hint.} Use polar coordinates to write 
$$  \int_{B_r(x)} \, f(y) \, dy \,=\, \int_0^r \int_{S^{n-1}} \, f(x+\rho z) \, d\sigma(z) \rho^{n-1} \, d\rho $$ where 
$\IS^{n-1}$ is the unite sphere in $\IR^n$

\vskip .1in
\noindent {\bf b)} Suppose now $f: \IR \times \IR^n \to \IR$, \, $f=f(r, x), \, r \in \IR$ and $x \in \IR^n$ (fixed). Let $$\phi(r):= \int_{B_r(x)} \, f(r, y) \, dy $$ and assume that $f$ and $\partial_r f$ are continuous.  Show then that 
$$ \frac{d}{dr} \phi (r) \,=\, \int_{B_r(x)} \, \partial_r f(r, y) \, dy  \,+\, \int_{\partial B_r(x)} \, f(r, y) \, d\sigma(y). $$
\noindent {\bf Hint.} Write  $$\dfrac{\phi(r+h) - \phi(r)}{h}$$ as $$ \int_{B_{r+h}(x)} \, \frac{f(r+h, y)  - f(r, y)}{h} \, dy  + \frac{1}{h} \biggl\{\int_{B_{r+h}(x)} \, f(r, y) dy   -  \int_{B_r(x)} f(r, y)\, dy \biggr\}. $$ Then use the Dominated Convergence Theorem on the first term and part {\bf a)} on the second term.


%\vskip .2in
%
%{\bf Additional Problems on Distributions.} 
%\vskip .2in
%
%(1).  Prove the Remark at the bottom of page 10 of the Notes on Distributions.  
%\vskip .2in
%
%A generalized version of this statement is the following problem: (how would you do it?)
% \vskip .1in
%
%(1') Let $\{ f_j\} \in L^1(\IR^n)$ be a sequence of  nonnegative functions such that 
%$$\align &\int_{\IR^n} \, f_j(x) \, dx \to 1 \quad \text{ as} \quad   j \to \infty \quad \text{ and for any }  a>0,  \\ 
%&\int_{|x| > a} \, f_j(x) \, dx \to 0 \quad \text{ as} \quad j \to \infty . \endalign$$Let $F_{f_j}$ be the distribution defined by $f_j$.  Prove that $F_{f_j} \to \delta$ in ${\Cal D}^{\prime}$. 
%
%\vskip .2in
%
%(2) Prove that the Dirac delta distribution  $\delta_{x_0}$  with point mass at $x_0 \in \IR^n$ (fixed) is not given by a locally integrable function.
%In other words prove that there does not exist any $f \in L^1_{\text loc} (\IR^n)$ such that 
%$$ \langle \delta_{x_0}, v \rangle = \langle f , v \rangle \qquad \text{ for all } \quad v \in C^{\infty}_0(\IR^n) $$
%
%\vskip .3in
%

%\centerline{ {\bf{ SET 6} }}
%\vskip .2in 
%
%{\bf Robert McOwen's Book Section 2.3:   \, 4, 8, 10, 11c).}
%
%\vskip .1in 
%
%{\bf Additional Problem}
%
% Compute $\frac{d}{dx} ( \log |x|) $  on $\IR$  in the sense of distributions. 
%\vskip .05in
%\noindent Recall that the principal value of $\frac{1}{x}$   ( pv  $\frac{1}{x}$ )  is defined as 
%$$ \langle  \text{pv} \frac{1}{x} , \phi \rangle = \text{pv} \int_{\IR} \frac{ \phi(x)}{x} \, dx := \lim_{\varepsilon \to 0^+} \int_{ |x| > \varepsilon} \,  \frac{ \phi(x)}{x} \, dx $$
%
%


\newpage


\centerline{ {\bf{ SET 5}  Due:  Thursday November 21, 2019}}
\vskip .2in 

{\bf Robert Owen's Book Section 4.1: \, \,  1, 2, 3, 5, 6}


\vskip .1in 

{\bf Robert Owen's Book Section 4.2:  \, \, 3, 4, 5, 6, 8, 10a) }



\vskip .3in


\centerline{ {\bf{ SET 6  Due:  Thursday December 5 }}}

\vskip .2in 

{\bf Robert Owen's Book Section 4.1: \, \,   7, 9 }

\vskip .1in

{\bf Robert Owen's Book Section 4.2:   \, \, 7, 11 }
%
\vskip .1in

{\bf Additional Problem}: Let $\Omega \subset \IR^n$, bounded open set and $g \in C^1(\partial \Omega)$.  Find the PDE for  $u \in C^2(\Omega)$ such that $u$ is the minimizer over $\A$ of $$ E(v) =\int_{\Omega} \sqrt{ 1 + | \nabla v|^2} \, dx,  $$ where 
$$ \A: = \{  v \in C^2(\Omega) \cap C^1(\partial \Omega) \,\, : \, \, v \equiv  g \, \text{ on } \partial \Omega\} $$


\vskip .3in
\centerline{ {\bf{ SET 7  Due:  Tuesday December 10th}}}

\vskip .2in



{\bf Robert Owen's Book Section 5.1: \, \,   2,  6,  7}

\vskip .1in
{\bf Robert Owen's Book Section 5.2:  \, \, 1, 2, 3, 4, 11 }
\vskip .3in

{\bf Additional Problems.} Compute (in the sense of distributions) the following Fourier transforms on $\IR^n$:
\vskip .1in 

(i) \, \, $\widehat{\delta_0}$  \qquad (ii) \, \, $\widehat{ D^{\alpha} \, \delta_0}$ \qquad (iii) \, \, $\widehat{ x^{\alpha}}$ \qquad (iv) \, \,  $\widehat{ H }$  \quad  where $H(x)$ is the Heaviside function.


\vskip .3in

\centerline{ {\bf{ EXTRA \,--\, Do these but do not turn in:} }}
\vskip .2in

{\bf Robert Owen's Book Section 6.1: \, \,  5a), 6, 15, 16}

\vskip .2in 

{\bf Robert Owen's Book Section 6.2: \, \, 1,  2, 4, 6}





\end{document}

\vskip .2in




\centerline{ {\bf{ SET 9  Due:  TBA }}}

\vskip .1in 
{\bf Robert Owen's Book Section 5.1:  2,  6,  7}

\vskip .1in
{\bf Robert Owen's Book Section 5.2:  1, 2, 3, 4, 11 }


\vskip .1in

{\bf Additional Problems.} Compute (in the sense of distributions) the following Fourier transforms on $\IR^n$:
\vskip .1in 

(i) \, \, $\widehat{\delta_0}$  \qquad (ii) \, \, $\widehat{ D^{\alpha} \, \delta_0}$ \qquad (iii) \, \, $\widehat{ x^{\alpha}}$ \qquad (iv) \, \,  $\widehat{ H }$  \quad  where $H(x)$ is the Heaviside function.
\vskip .3in

\centerline{ {\bf{ EXTRA \,--\, Do these but do not turn in:} }}
\vskip .2in

{\bf Robert Owen's Book Section 6.1: \, \,  5a), 6, 15, 16}

\vskip .2in 

{\bf Robert Owen's Book Section 6.2: \, \, 1,  2, 4, 6}



\vskip .2in

\centerline{ {\bf{ SET 10 \,--\, Do these but do not turn in:} }}
\vskip .2in

{\bf Robert Owen's Book Section 6.1: \, \,  5a), 6, 15, 16}

\vskip .2in 

{\bf Robert Owen's Book Section 6.2: \, \, 1,  2, 4, 6}


\enddocument



                                               













%%%%%%%%%%%%%%  FALL 2011 %%%%%%%%%%%%%%%%%%%%%%%%%%%%%%%%

%%%%%%%%%%%%%%% FAL 2011   %%%%%%%%%%%%%%%%%%%%%%%%%%%%%%%%


\centerline{ {\bf{ SET 6} }}
\vskip .1in 

{\bf Robert McOwen's Book Section 2.3:   \, 4, 8, 10, 11c).}
\vskip .1in 

{\bf Additional Problems.}

\vskip .1in 

(1) Let $\{ f_j\} \in L^1(\IR^n)$ be a sequence of  nonnegative functions such that 
$$\align &\int_{\IR^n} \, f_j(x) \, dx \to 1 \quad \text{ as} \quad   j \to \infty \quad \text{ and for any }  a>0,  \\ 
&\int_{|x| > a} \, f_j(x) \, dx \to 0 \quad \text{ as} \quad j \to \infty . \endalign$$Let $F_{f_j}$ be the distribution defined by $f_j$.  Prove that $F_{f_j} \to \delta$ in ${\Cal D}^{\prime}$. 

\vskip .2in 


(2)  Compute $\frac{d}{dx} ( \log |x|) $  on $\IR$  in the sense of distributions. 
\vskip .05in
\noindent Recall that the principal value of $\frac{1}{x}$   ( pv  $\frac{1}{x}$ )  is defined as 
$$ \langle  \text{pv} \frac{1}{x} , \phi \rangle = \text{pv} \int_{\IR} \frac{ \phi(x)}{x} \, dx := \lim_{\varepsilon \to 0^+} \int_{ |x| > \varepsilon} \,  \frac{ \phi(x)}{x} \, dx $$


\vskip .2in




\centerline{ {\bf{ SET 5} }}
\vskip .1in


{\bf Robert McOwen's Book Section 3.3:   \, 1, 2, 4, 5. }
\vskip .1in 
{\bf Additional Problem.} Let $f: \IR^n \to \IR$ be a continuous function and let $x\in \IR^n$ be fixed. For $r >0$ let 
$$  B_r(x):=\, \{ \, y \in \IR^n \, :\, | x-y | \, \leq \, r\, \}, \quad\text{and}\quad \partial B_r(x):= \{ \, y \in \IR^n \, :\, | x-y | \, =\, r\, \}.$$ {\bf a)} Prove that $$ \frac{d}{dr} \int_{B_r(x)} \, f(y) \, dy \,=\, \int_{\partial B_r(x)} \, f(y) \, d\sigma(y) $$ where $d\sigma$ is the surface measure. 

\noindent {\bf Hint.} Use polar coordinates to write 
$$  \int_{B_r(x)} \, f(y) \, dy \,=\, \int_0^r \int_{S^{n-1}} \, f(x+\rho z) \, d\sigma(z) \rho^{n-1} \, d\rho $$ where 
$\IS^{n-1}$ is the unite sphere in $\IR^n$

\vskip .1in
\noindent {\bf b)} Suppose now $f: \IR \times \IR^n \to \IR$, \, $f=f(r, x), \, r \in \IR$ and $x \in \IR^n$ (fixed). Let $$\phi(r):= \int_{B_r(x)} \, f(r, x) \, dy $$ and assume that $f$ and $\partial_r f$ are continuous.  Show then that 
$$ \frac{d}{dr} \phi (r) \,=\, \int_{B_r(x)} \, \partial_r f(r, x) \, dy  \,+\, \int_{\partial B_r(x)} \, f(r, x) \, d\sigma(y). $$
\noindent {\bf Hint.} Write  $$\dfrac{\phi(r+h) - \phi(r)}{h}$$ as $$ \int_{B_{r+h}(x)} \, \frac{f(r+h, x)  - f(r, y)}{h} \, dy  + \frac{1}{h} \biggl\{\int_{B_{r+h}(x)} \, f(r, x) dy   -  \int_{B_r(x)} f(r, y)\, dy \biggr\}. $$ Then use the Dominated Convergence Theorem on the first term and part {\bf a)} on the second term.


\vskip .2in


\centerline{ {\bf{ SET 4} }}
\vskip .1in
{\bf Robert McOwen's Book Section 3.2:   \, 1, 2, 2*, 3, 5, 6a)}

\vskip .1in 
{\bf Hint for Problem 2} First show that if $y=(y_1, y_2, y_3)$  is a point in the unit sphere $\IS^2$ then $$ \int_{\IS^2} \, y_j \, d\sigma(y) \, =\, 0 \qquad  \quad j=1, 2, 3$$ where as always $d \sigma$ is the area surface element. This can be proved by an explicit calculation using spherical coordinates or also by symmetry, splitting for each $j$, the integral over the sphere $\IS^2$ into the two integrals for the half spheres $y_j \geq 0$ and $y_j \leq 0$ and showing the two integrals cancel out).

\vskip .1in 

{\bf Problem 2*} Same as problem 2 but with initial conditions $u(x,0)=0$ and $u_t(x,0) = x_2$. 
\vskip .2in 
{\bf Hint for Problem 5} Follow the hint in the back of the book and define $u(x_1, x_2, x_3) = \cos(\frac{m}{c} x_3) v (x_1, x_2, t)$. Then prove by a direct calculation $u$ satisfies the $3d$ (linear homogeneous) wave equation. Hence $u$ can be represented in terms of $g$ and $h$ using Kirchhoff's formula. Assume first for simplicity that $g=0$ and write the explicit formula for $u$ in this case. Then set $x_3=0$ and proceed as in the `method of descent' (parametrize the two halves of  $\IS^2$ corresponding by graphs   $y_3= \pm \sqrt{1 -(y_1^2 + y_2^2)}$) to 
obtain a formula for $v$ which should look like: 
$$ v (x_1, x_2, t) = C \, t \, \int_D \, \frac{ \cos\biggl( m t \sqrt{ 1 -(y_1^2 + y_2^2)}\biggr) \, h(x_1 + c t y_1, x_2 + ct y_2) }{  \sqrt{ 1 -(y_1^2 + y_2^2)} } \, dy_1 dy_2 $$ where $D:=\{ (y_1, y_2) : \, y_1^2 + y_2^2 \leq 1\, \}$.

Finally drop the assumption that $g=0$ to get the general formula. 

\vskip .2in 

{\bf Hint for Problem 6a)} The decay in $t$ in $3d$ is due to waves spreading out in space on expanding spheres $\partial B(x, ct)$ as $t \to \infty$. This bound reflects the {\it dispersion} of waves.  Assume first $g=0$ and use Kirchhoff's formula in conjunction with the fact that $h$ is bounded (i.e. \,  $|h(x)| \leq M$ for some $M>0$) and compactly supported ( hence $h(x) =0$ if $|x| \geq R$ for some large $R>0$). From these (plus change of variables) one can prove that 
$$| u(x, t)| \leq \frac{M}{ C t} \text{area}(\partial B(x, ct) \cap B(0, R))$$
where $C>0$ depends on the area of the unit sphere and the speed $c$. Next show that the area of the {\it spherical cap} $\partial B(x, ct) \cap B(0, R)$ can be bounded by some absolute constant times $R^2$- and hence by a quantity that is independent of $x$ and $t$.  Next note that if $h=0$ instead then both terms in $g$ in Kirchhoff's formula can be treated similarly as the previous case (do it!). Finally, for $g, h \in C^{\infty}_0(\IR^3)$ we have the sum of 3 terms all of which we know how to treat. 


\vskip .1in 
{\bf Note related to part 6b)}. In $2d$ there is `one less direction' than in $3d$ for waves to spread out so intuitively we expect the amplitude of the waves to decay slower as time increases.  And indeed, in $2d$ there is decay bound of the form $| u(x, t)| \leq \frac{C}{\sqrt{t}}$ but this is harder to prove.  In $1d$ , however, as we can clearly see from D'Alembert's formula there is no decay at all as $t \to \infty$. 

\vskip .2in 

\centerline{ {\bf{ SET 3} }}

\vskip .1in

{\bf Robert McOwen's Book Section 3.1:   \, 1, 4}  

\vskip .1in

{\bf Additional Problems.} 
\vskip .1in 
(1)  (a)Show that the general solution to the PDE $u_{xy}=0$  is
$$ u(x,y) = F(x) + G(y)$$ 

for arbitrary functions $F,\, G$
\vskip .1in
(b) Using a change of variables $\xi = x+t$ and $\eta = x -t$, show that $$ u_{tt} - u_{xx} =0 \quad \text{ if and only if } \quad u_{\xi \eta} = 0$$
\vskip .1in
(c) Use parts (a) and (b) to rederive D'Alembert's formula.

\vskip .2in 

(2) Let $u \in C^2(\IR \times [0, \infty))$ be a solution to the Cauchy initial value problem  $$\cases u_{tt} - u_{xx} = 0  \qquad \text{in} \quad \IR \times (0, \infty)\\
u(0, x) = g(x)  \qquad \text{in} \quad \IR \\
u_t(0, x) = h(x)  \qquad \text{in} \quad  \IR  \endcases $$ Suppose that $g$ and $h$ are smooth and have compact support.  The {\it kinetic energy} is $$k(t):= \frac{1}{2} \int_{\IR} \, u_t^2(t, x) \, dx $$ and the {\it potential energy } is 
$$p(t):= \frac{1}{2} \int_{\IR} \, u_x^2(t, x) \, dx $$ 
Prove : 

(a) k(t) + p(t) is constant in $t$.
\vskip .1in 
(b)  k(t) = p(t) for all {\it large enough} times $t$.
 
\vskip .1in



\vskip .2in

\centerline{ {\bf{ SET 2} }}

\vskip .1in

{\bf From McOwen's Book Section 2.1:\quad  1, 2, 7 (modified in class). } 

\vskip .1in

{\bf From McOwen's Book Section 2.2:} \, Read examples in 2.2a); read all of Section 2.2b). Then  {\bf do \, $1$}.


\vskip .2in




%%%%%%%%%%%%%%%%%%%%%%%  OLDER %%%%%%%%%%%%%%
%%%%%%%%%%%%%%%%%%%%%%%  OLDER %%%%%%%%%%%%%%%




\enddocument



                                               


\centerline{ {\bf{ SET 9} }}
\vskip .1in

{\bf Robert Owen's Book Section 6.2: \, \, 1,  2, 4, 6}

\vskip .2in
{\bf Robert Owen's Book Section 6.1: \, \,  5a), 6, 15, 16}

\vskip .2in 

\centerline{ {\bf{ SET 8} }}
\vskip .1in
{\bf Robert Owen's Book Section 5.2:  1, 2, 3, 4, 11 }

\vskip .1in
{\bf Robert Owen's Book Section 5.1:  2, 6, 7}

\vskip .1in

{\bf Additional Problems.} Compute (in the sense of distributions) the following Fourier transforms on $\IR^n$:
\vskip .1in 

(i) \, \, $\widehat{\delta_0}$  \qquad (ii) \, \, $\widehat{ D^{\alpha} \, \delta_0}$ \qquad (iii) \, \, $\widehat{ x^{\alpha}}$ \qquad (iv) \, \,  $\widehat{ H }$  \quad  where $H(x)$ is the Heaviside function.
\vskip .2in

\centerline{ {\bf{ SET 7} }}
\vskip .1in 
{\bf Additional Problem}: Let $\Omega \subset \IR^n$, bounded open set and $g \in C^1(\partial \Omega)$.  Find the PDE for  $u \in C^2(\Omega)$ such that $u$ is the minimizer over $\A$ of $$ E(v) =\int_{\Omega} \sqrt{ 1 + | \nabla v|^2} \, dx $$ Here $\A$ is the as we defined in class.

\vskip .2in

{\bf Robert Owen's Book Section 4.2:  3, 4, 5, 6, 8, 10a) }

\vskip .1in
{\bf Robert Owen's Book Section 4.1:  1, 2, 5, 6, 7 }


\vskip .2in 



\centerline{ {\bf{ SET 6} }}
\vskip .2in 
{\bf Robert Owen's Book Section 2.3:   \, 4, 8, 10, 11c).}
\vskip .1in 

{\bf Additional Problems.}

\vskip .1in 

(1) Let $\{ f_j\} \in L^1(\IR^n)$ be a sequence of  nonnegative functions such that 
$$\align &\int_{\IR^n} \, f_j(x) \, dx \to 1 \quad \text{ as} \quad   j \to \infty \quad \text{ and for any }  a>0,  \\ 
&\int_{|x| > a} \, f_j(x) \, dx \to 0 \quad \text{ as} \quad j \to \infty . \endalign$$Let $F_{f_j}$ be the distribution defined by $f_j$.  Prove that $F_{f_j} \to \delta$ in ${\Cal D}^{\prime}$. 

\vskip .2in 


(2)  Compute $\frac{d}{dx} ( \log |x|) $  on $\IR$  in the sense of distributions. 
\vskip .05in
( Hint:  recall that the principal value of $\frac{1}{x}$ is defined as 
$$ \langle  \text{pv} \frac{1}{x} , \phi \rangle = \text{pv} \int_{\IR} \frac{ \phi(x)}{x} \, dx := \lim_{\varepsilon \to 0^+} \int_{ |x| > \varepsilon} \,  \frac{ \phi(x)}{x} \, dx $$

\vskip .2in




\centerline{ {\bf{ SET 4} }}
\vskip .1in
{\bf Robert Owen's Book Section 3.2:   \, 1, 2, 2*, 3.}

\vskip .1in 
{\bf Problem 2*} Same as Problem 2 but with initial conditions to $u(x,0)=0$ and $u_t(x,0) = x_2$ instead. 

\vskip .1in
{\bf Additional Problem.}

Let $u$ solve $$\cases u_{tt} - \Delta u &= 0 \qquad  \text{ in } \IR^3
\times (0, \infty) \\
u = g, \quad u_t &= h \qquad \text{ on }  \IR^3 \times \{ t=0\}
\endcases $$
  
where $g$ and $h$ are smooth functions with \underbar{compact
support}. Show that there exists a constant $C>0$ such that 
$$ | u(x, t) | \le \frac{C}{t} \qquad \text{ for any } \, x \in \IR^3,
t>0 $$

\vskip .2in 


\centerline{ {\bf{ SET 3} }}

\vskip .1in

{\bf Robert Owen's Book Section 3.1:   \, 1, 4,  8.}  

\vskip .1in

{\bf Additional Problems.} 
\vskip .1in 
(1)  (a)Show that the general solution to the PDE $u_{xy}=0$  is
$$ u(x,y) = F(x) + G(y)$$ 

for arbitrary functions $F,\, G$
\vskip .1in
(b) Using a change of variables $\xi = x+t$ and $\eta = x -t$, show that $$ u_{tt} - u_{xx} =0 \quad \text{ if and only if } \quad u_{\xi \eta} = 0$$
\vskip .1in
(c) Use parts (a) and (b) to rederive D'Alembert's formula.

\vskip .2in 

(2) Let $u \in C^2(\IR \times [0, \infty))$ solve the initial value problem  $$\cases u_{tt} - u_{xx} = 0  \qquad \text{in} \quad \IR \times (0, \infty)\\
u(0, x) = g(x)  \qquad \text{in} \quad \IR \\
u_t(0, x) = h(x)  \qquad \text{in} \quad  \IR  \endcases $$ Suppose that $g$ and $h$ are smooth and have compact support.  The {\it kinetic energy} is $$k(t):= \frac{1}{2} \int_{\IR} \, u_t^2(t, x) \, dx $$ and the {\it potential energy } is 
$$p(t):= \frac{1}{2} \int_{\IR} \, u_x^2(t, x) \, dx $$ 
Prove : 

(a) k(t) + p(t) is constant in $t$.
\vskip .1in 
(b)  k(t) = p(t) for all {\it large enough} times $t$.
 
\vskip .1in


\centerline{ {\bf{ SET 2} }}

\vskip .1in

{\bf Robert Owen's Book Section 2.1:\quad  1, 2, 7. } 

\vskip .1in

{\bf Robert Owen's Book Section 2.2:} \, Read examples in 2.2a); read all of Section 2.2b), and then do  $1$.


