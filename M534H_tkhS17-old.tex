\magnification=1100
\input amstex

\documentstyle{amsppt}
\NoBlackBoxes
\NoPageNumbers
\pagewidth{16truecm}
\pageheight{22truecm}
\vcorrection{-1.truecm}

\def \IR {\Bbb R}
\def \IN {\Bbb N}
\def \IC {\Bbb C}
\def \B {\Cal B}
\def \M {\Cal M}
\def \N {\Cal N}
\def \H {\Cal H}
\def \C {\Cal C}
\def \R {\Cal R}
\def \A {\Cal A}
{\catcode`@=11
\gdef\nologo{\let\logo@\empty}
\catcode`@=12}
\nologo
%\hcorrection{.3in}
\hskip 3.2in {\bf NAME: }
\vskip .15in 
\hskip 3.2in {\bf  ID \#: }

\vskip .7in

\centerline{ \bf TAKE HOME FINAL MATH 534H}
\vskip .2in
\centerline{ Saturday May 6th,  2017 }
\vskip 1in
\document
\subhead{ Instructions }\endsubhead
\vskip .3in 
 
\roster


\item {\bf This exam consists of 4 problems with parts for a total of $\bold{100\%}$.}
\vskip .3in 

\item {\bf  It is due no later than Tuesday May 9th by {\bf 3:30 pm}  in LGRT 1338 } 
\vskip .3in 

\item {\bf You should work on it alone. You may consult Salsa'  book,  the class notes and your homework ONLY. \, No other material is allowed. }

\vskip .3in 

\item {\bf Show \underbar{all} the work needed to reach your answer for full credit.}
\vskip .3in 

\item {\bf You cannot discuss the problems with other people, including
classmates.}
\vskip .2in 


\item {\bf \underbar{Type} each problem and its solution in an ordered fashion (new page for each problem)  and staple them all together with this cover. 
Insert additional pages if needed.}

\endroster 



\newpage 


{\underbar{Final Problem 1}:  Find the solution $u(x,t)$ to the following inhomogeneous diffusion boundary/initial value problem  with Dirichlet boundary conditions (proceed as in handout example). 

\vskip .05in
  $$\cases u_{t} \,- \,  u_{xx} \, =\, e^{-t},   \qquad  0< x < 1, \, \, \, t>0  \\
  u(x, 0) \, =\, 1, \quad 0< x<1  \\
u(0, t) \,= \, 0 \quad \text{and} \quad u(1, t) \,=\, 0 \endcases $$

\underbar{Hints}  First find the sine Fourier series for $e^{-t}$ on $(0, 1)$. Note that you had to find the sine Fourier series for $1$ in Set  2 Problem 4.  

At some point you'll encounter an ODE of the form  $$a_n^{\prime}(t)  + c_1 n^2 \pi^2  a_n(t)  = c_2 \frac{e^{-t}}{n\pi}$$  for some specific constants $c_1, c_2$. To find the solution to this ODE, consider $a_n(t) = A_n e^{-t}$ for suitable $A_n$ that you would need to find.

\bigskip

\newpage


{\underbar{Final Problem 2}}:  a) Let $f$ be a smooth function with compact support (that is  the support of f is contained in some fixed ball of radius $R.0$ centered at the origin which means,  $f(x) =0$ for $|x| > R$ for some $R>0$ ). Consider the Poisson equation on the whole space  $\Bbb R^d$ 
$$ (\dagger)  \qquad  \Delta u = f  $$
\bigskip 

i)  Consider $d=3$. We know that  there exist smooth solutions $u$ that tend to zero at $\infty$  ($u(x)  \to  0$ as $|x| \to \infty$).  \, Use Liouville's theorem to prove that these solutions are unique. Carefully show all steps and justify them. 
 
 \bigskip
 
ii) Consider $d=2$.  We know  now that there exist smooth solutions solutions $u(x)$ which could grow like $\ln |x|$ at infinity.   \, Use Liouville's theorem to prove that smooth solutions $u(x)$ whose gradient $|\nabla u(x)|  \to  0$ as $|x| \to \infty$  are unique up to a constant;  \, i.e. if $u_1$ and $u_2$ are solutions, then $u_1 = u_2 + C$ for some $C>0$. Carefully show all steps and justify them. 
 

\vskip .5in

b) Let $\Omega$ be a smooth  and bounded domain in $\Bbb R^d$ and let $g$ be a given smooth function on $\partial \Omega$,  the boundary of $\Omega$. Consider  the Dirichlet boundary value problem for the Laplace equation:
$$ \cases \, &\Delta u=\, 0 \qquad \qquad x \in \Omega \\
&u( {\overline x}) \, = \, g({\overline x}) \qquad  {\overline x} \in \partial \Omega
\endcases $$
 Use the energy method to show that smooth (and continuous up to the boundary) solutions  to this problem are unique. To prove this consider $w = u_1 - u_2$  (where $u_1, u_2$ are two solutions)  and write down the boundary value problem that $w$ solves.  Next, prove that 
 $$ E(w) = \int_{\Omega}  | \nabla w (x) |^2 \, dx \, =\, 0 \tag1$$
 and use this in conjunction with the boundary values of  $w$ --and the fact that $w$ is continuous on $\overline{\Omega}$ --  to conclude uniqueness.
 
 \underbar{Hint} To prove (1), note that $$ 0 = \int_{\Omega}  w(x) \, \Delta w(x)\,  dx $$ and integrate by parts/divergence theorem.

\newpage


{\underbar{Final Problem 3}}

  Consider the solution to the initial value problem for the wave equation on $\IR$:
$$ \cases u_{tt} - u_{xx} \,=\, 0  \qquad x \in \IR, \\
u(x, 0) = 0 \\
u_t (x, 0) = \chi_{[-2, 2]}(x)  \quad \text{this is }  1 \text{ on }  |x| \le 2; \text{ and } \, 0 \, \text{ otherwise. } \endcases $$ 


(a) Use D' Alembert's formula to write down the solution $u(x,t)$ in terms of its initial data (do not evaluate the integral).
  \vskip .2in

(b) Use differentiation rules for integrals to compute $u_t (x,  t)$.

\noindent (\underbar{Hint} Recall homework problem 8 in Set 4). 
  \vskip .2in
(c)  Set $x=0$ in (b) and prove that for all $|t|>2$ we have that $u_t (0,  t) \,=\, 0$. 

\newpage

{\underbar{Final Problem 4}}:

Consider the following linear wave equation equation on the interval $[0,\pi]$ with zero Dirichlet boundary conditions:

$$ \cases \, &u_{tt} \, - \, u_{xx} \,=\, 0 \qquad \qquad  0 < x < \pi \\
&u(0, t) \, = \, 0 = u(\pi, t) \\
& u(x, 0)= \phi(x)  \qquad u_t(x, 0) = \psi(x) \endcases \tag{1}$$ where $\phi, \psi$ are smooth functions on $[0, \pi]$

a) Use separation of variables and fully solve the corresponding eigenvalue problems associated to (1) to write down the sine-cosine series expansion of the solution $u(x,t) \, =\, \sum_{n=1}^{\infty} \, X_n(x) T_n(t)$.  Describe the coefficients in terms of the initial data $\phi(x)$ and $\psi(x)$.   

  \vskip .2in
  
  b) Suppose that $\phi(x) =x$ and $\psi(x)=0$.  Find the sine Fourier series for  $\phi$ in this case and use it to give the explicit solution to $(1)$ by finding the corresponding coefficients in a).
  
  


\enddocument