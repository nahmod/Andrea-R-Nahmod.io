\magnification=1200
\input amstex

\documentstyle{amsppt}
\NoBlackBoxes
%\NoPageNumbers
\def \IR {\Bbb R}
\def \IN {\Bbb N}
\def \IC {\Bbb C}
\def \B {\Cal B}
\def \M {\Cal M}
\def \N {\Cal N}
\def \H {\Cal H}
\def \C {\Cal C}
\def \R {\Cal R}
\def \A {\Cal A}
{\catcode`@=11
\gdef\nologo{\let\logo@\empty}
\catcode`@=12}
\nologo
%\hcorrection{.3in}
\hskip 3in {\bf NAME: }
\vskip .2in
\centerline{ \bf Real Analysis II, Final Exam}
\vskip .1in
\centerline{May 17th 2009}
\vskip .2in
\document

\subhead{ Instructions }\endsubhead
\vskip .2in 
 
{\bf { \roster


\item This exam is due no later than Thursday May 21 at 4pm.   

\vskip .2in 

\item  It has 4 questions  for a total of $100$ points. 
\vskip .2in 

 
\item You must work on this exam alone. 
\vskip .2in 

\item  You may consult Folland's book or the class notes \underbar{ONLY}. \newline Absolutely \underbar{no other books or notes} are permitted.
\vskip .2in 

\item You cannot discuss the problems with other people, including
classmates.
\vskip .2in 

\item You may ask me any questions you have.
\vskip .2in 

\item  {\bf State explicitely all results that you use in your proofs and
verify that these results apply.}


\vskip .2in 

\item Show all your work and \underbar{ justify each and all steps} in your proofs. 

\vskip .2in 

\item Write each problem and its solution following each problem and staple them all together with this cover. 
Insert additional pages if needed.
\endroster } 




\vskip .3in 


\subhead{Conventions and Definitions }\endsubhead 
\vskip .2in 

\roster 

\item For a set $A$, \, $\chi_A$ denotes the characteristic function of $A$. 
\vskip .2in 

\item If a measure is not specified, use the Lebesgue measure on
$\IR^n$. This measure is denoted by $m$.  If a $\sigma$-algebra on $\IR^n$ is not specified, use the
Borel $\sigma$-algebra.


\endroster}


\newpage 


${\bold 1}$)  Let $\H$ be a Hilbert space with inner product  $\langle
\cdot , \cdot \rangle $
\vskip .2in 
(a)  Suppose that $\H$ is a {\it real} Hilbert space; that is suppose
that  $\langle \cdot , \cdot \rangle: \H \times \H \to \IR.$ 
Show then that $x \perp y$ if and only if the identity $\Vert
x+y\Vert^2 =\Vert x\Vert^2 +\Vert y\Vert^2 $ holds. 

\vskip .15in 

(b)  Suppose now that $\H$ is a {\it complex} Hilbert space; that is suppose
that  $\langle \cdot , \cdot \rangle : \H \times \H \to \IC$ . 
Show with an example that the identity   $\Vert x+y\Vert^2 =\Vert x\Vert^2 +\Vert
y\Vert^2$  may \underbar{not} imply that  $x \perp y$ . 

\vskip .15in






(c) Let $\{x_n\}_{n \ge 1} $ be an {\it orthonormal basis} of $\Cal H$ and define
$$ y_n := x_{n+1} - {x_n} \qquad \qquad n \geq 1. $$ Show that if $z$
is orthogonal to $y_n$ for each $n \ge 1$, then $z=0$. 

\vskip .15in 



(d) Use Bessel's inequality  to show that if  $\{x_n\}_n $ is any orthonormal sequence in $\Cal H$, 
then $$ \lim_{n \to \infty} \langle x_n , y \rangle =0 \qquad \text{ for
all } \, y \in \H. $$ 

\vskip .2in 
(e) Let $\H = L^2([0, 2 \pi), m )$.
\vskip .2in 



\hskip .2in (i) Prove that the system $\{ \sqrt{1/\pi} \cos(kx) \, : \, k \in \IN\} $ is 
orthonormal in $L^2([0, 2\pi),m)$. 


\vskip .2in 


\hskip .2in (ii) Let $f$ be any function in $L^2([0, 2 \pi),m)$. 
Prove that $$\lim_{k \to \infty} \int_0^{2 \pi} \, f(x) \,
\cos(kx)   \, dx \, = \, 0 $$





\newpage


 
${\bold 2}$)  Let $\H$ be a Hilbert space and let $M$ be a non-trivial closed subspace of $\H$.  

\noindent Let $P_{M}: \H \to \H$ be the orthogonal projection from $\H$ to $M$. You may assume without proof that $P_M$ is linear and in the notation of Th. 5.24, that $P_{M} x = y \in M$ and that $x- P_Mx= z \in M^{\perp}$ for all $x \in H$. 

\vskip .1in 

(a) \, Show that  $P_M$ is bounded and that moreover, $\Vert P_M  \Vert_{op} = 1$ where 
$$\Vert P_M  \Vert_{op} := \sup \{ \Vert P_M x \Vert_{\H} \, : \, \Vert x \Vert_{\H} \le 1\, \}.$$

\vskip .2in 

(b) \,  Show that $P_{M} (P_{M} x) =  P_M x$ for all $x \in \H$; that is that $P^2_M=P_M$

\vskip .2in

(c) \, Show that  $P_M$ is self-adjoint. That is, using the notation of Pb57 in \&5.5, show that $P_M x=P_M^{\ast} x $ for all $x \in \H$. 


\newpage 

${\bold 3}$)  \, Let $V$ be a vector space and $\A$ a non-empty index set. For each $\alpha \in \A$ let $Y_{\alpha}$ be a subspace of $V$ equipped with a norm $\Vert \cdot \Vert_{Y_{\alpha}}$ which we assume turns $Y_{\alpha}$ into a Banach space for each $\alpha \in \A$.  Furthermore, assume the following property holds:

\vskip .2in

\quad  If $\alpha, \beta \in \A$ and a sequence $ x_n \in Y_{\alpha} \cap Y_{\beta}$ converges to $x$ in the $Y_{\alpha}$-norm and 

\quad to $x'$ in the $Y_{\beta}$-norm  then $x=x'$. 

\vskip .2in 

\noindent Define a new vector space $X$ by $$ X:= \{ x \in \bigcap_{\alpha \in \A} Y_{\alpha} \, :\, \sum_{\alpha \in \A} \Vert x \Vert_{Y_{\alpha}}  < \infty \, \}$$ and equip this space with the norm defined by 
$$ \Vert x \Vert_{X}:=  \sum_{\alpha \in \A} \Vert x \Vert_{Y_{\alpha}} $$

\noindent Show:  (a)  that  $ \Vert x \Vert_{X}$ is indeed a norm and (b) that this norm turns $X$ into a Banach space.



\newpage 
${\bold 4}$)  Let $1 \le p < \infty$ and  $\{f_n\}_{n\ge 1}$ a sequence in $L^p(\IR^d)$.  

\vskip .1in 

(a) \,  Suppose there is an $f \in L^p(\IR^n)$ such that $f_n \to f$ in $L^p(\IR^d)$.  Prove that there exists constants $N_0 \in \IN$ and $C_p>0$ such that $\Vert f_n\Vert_p  \le C_p  \Vert f\Vert_p$ for  $n \ge N_0$.  

\vskip .2in 

(b) \, {\underbar{Use H\"older's inequality}} and part (a) to prove the following:  
\vskip .1in 

If $\{f_n\}$ is Cauchy in $L^p$ then there exists $f \in L^p(\IR^d)$ such that $$\int_{\IR^d}\, |  \, |f(x)|^p  -  | f_n(x)|^p \,| \, dx \to 0 \qquad \text{ as } \qquad n \to \infty$$

\vskip .2in

Hint.  Before applying H\"older,  note that  for any $a, b >0$ and $p\ge 1$ $$|a^p - b^p| \le p  \, \max{\{ a, b\} }^{p-1}  |a-b| \le p \,  (|a| + |b|)^{p-1} \, |a-b|$$  which is easily proved using the mean value theorem on $h(x)= x^p, \, x>0$.  

\enddocument 

${\bold 3}$)   (a) \, For any $1 \le p < \infty$ we know that $L^p$ is a Banach space. So, for  any   \underbar{{\it  fixed}} \,  $ z \in \IR$  we define the operator $T_z : L^p(\IR) \to L^p(\IR)$ by  $$ T_z(f)(x) \, = \, f(x -z) , \qquad \qquad \text{ for all} \quad x \in \IR $$  

(i)  Show that  $T_z \in {\Cal L} ( L^p, L^p)$,  i.e that $T_z$ is linear and bounded on $L^p(\IR)$. 

  
\vskip 1in   

(ii)   Show that $T_z$  is an invertible operator on $L^p$  and compute  $\Vert T_z^{-1} \Vert_{op}  $    

             
\vskip 1in 


(b)  Show that for any $g \in C_c(\IR)$,  and  any $1 \le p < \infty$, $$\lim_{y\to 0}  \Vert T_{z+y} (g) \, - \, T_z (g) \Vert_{L^p(\IR)} \, = \,  0$$

{\bf Hints.}  Note that $T_{z+y} = T_y \circ T_z$ and use it to reduce first to $z=0$ relying on part (a(i)).  

                    Show that if $g \in C_c$ then $T_y(g)$ are all supported in a common compact provided $|y| \le 1$. 
                    
                     Use the Heine-Borel theorem ( "Any continuous function with compact support is  uniformly continuous on its support. ") 
                     



