\magnification=1200
\input amstex

\documentstyle{amsppt}
\NoBlackBoxes
%\NoPageNumbers
\pagewidth{16.5truecm}
\pageheight{22truecm}

{\catcode`@=11
\gdef\nologo{\let\logo@\empty}
\catcode`@=12}
\nologo
%\hcorrection{.3in}

\centerline{ \bf  M623 HOMEWORK  -- Fall 2016}
\vskip .1in
\centerline{ Prof. Andrea R. Nahmod }
\vskip .2in

\document


\head {\underbar{SET 3:  Due Date 10/20/16}  }\endhead

\vskip .1in


\noindent {\bf From Chapter 1 (pp 37-42)}:\,  2, 6,  10, 11,  17, 22.

\underbar {Hint for $17$}:  Note that $$ \{ x : |f_n(x)| \, =\, \infty \} =  \bigcap_{j=1}^\infty \{ x : |f_n(x)| > \frac{j}{n} \, \}.$$ Hence the hypothesis implies that for each $n$, 
$$m ( \bigcap_{j=1}^\infty \{ x : |f_n(x)| > \frac{j}{n} \, \}) \, =\, 0. $$ But then $\lim_{j \to \infty} m (\{ x : |f_n(x)| > \frac{j}{n} \, \}) \, =\, 0$ (Why? Justify.).
Next, follow the hint in the book from here.
\bigskip

\underbar{Bonus Problem$^{\ast}$}: Do  32 (pp 44-45). Note that part b) should read  ``...prove that {\it there exists} a subset of G which is....".
\bigskip

\underbar{Additional Problems--Chapter I} (to do)

\medskip


{\bf I.}   a) \, Let $ A = \cup_{n=1}^{\infty} A_n$ with $ m_{\ast}(A_n) = 0$. \underbar{Use} the definition of exterior measure to prove that $m_{\ast}(A) =0$.

\smallskip

b)\, Use a) to prove that any countable set in $\Bbb R^d$ is measurable and has measure zero. 


\bigskip


{\bf II.}  Let $\{E_n\}_{n\geq 1}$ be a countable collection of measurable sets in $\Bbb R^d$. Define
$$  \quad  \limsup_{n \to \infty} E_n \,:=\, \{ x \in \Bbb R^d \, :\, \, x \in E_n, \, \text{ for infinitely many } \, n \, \} $$
$$  \quad  \liminf_{n \to \infty} E_n \,:=\, \{ x \in \Bbb R^d \, :\, \, x \in E_n, \, \text{ for all but finitely many } \, n \, \} $$
 
\smallskip 

a) Show that  $$\limsup_{n\to \infty} E_n \,=\, \bigcap_{n=1}^\infty \bigcup_{k=n}^{\infty} E_k \qquad \liminf_{n\to \infty} E_n \,=\, \bigcup_{n=1}^\infty \bigcap_{j=n}^{\infty} E_j $$

b) Show that $$ m( \liminf_{n\to \infty} E_n ) \leq \liminf_{n\to \infty} m(E_n ) $$ 
$$  m( \limsup_{n\to \infty} E_n ) \geq   \liminf_{n\to \infty} m(E_n )  \quad \text{ provided that } \, m(\bigcup_{n=1}^\infty E_n ) < \infty $$


\bigskip

{\bf III.} The following relates to the proof {\bf given in class} for Theorem 4.1  page  . Prove that the sequence of nonnegative simple functions $\{\phi_k\}_k$ that 
approximate pointwise $f$ is indeed increasing, ie. $\phi_k  \leq \phi_{k+1}$


\vskip .2in

\head {\underbar{SET 4:  Due Date 10/27/16}  }\endhead

\smallskip

\noindent {\bf From Chapter 2 (pp 89-97)}:  6,  9, 10,  11.

\smallskip
\noindent {\bf From Chapter 2 (pp 88-97)}: 2, 8, 15 


\vskip .15in
\underbar{Additional Problems:}
\medskip

{\bf I.\,}  Fill in all details to give a full proof of  Lemma 1.2 (ii)  (Chapter 2 pages 53-54).


{\bf II.\,} If a function $f$ is integrable then we proved in Proposition 1.12 (Chapter 2) that for any $\varepsilon>0$ there exists a $\delta >0$
such that for any set $A$ with $m(A) \leq \delta$, we have that $\int_A |f(x)| \, dm \leq \varepsilon$ ({\it absolute continuity of the Lebesgue integral}).

\smallskip

 We say that a sequence of functions $\{ f_n\}_{n\geq 1}$ is {\bf equi-integrable} if for every $\varepsilon >0$
there exists $\delta >0$ s.t.  for any set $A$ with $m(A) \leq \delta$, we have that $\int_A |f_n(x)| \, dm \leq \varepsilon$ \underbar{for all} $n \geq 1$. 

\smallskip

Now prove the following.   

\smallskip

Let $E$ be  a set of finite measure, $m(E) < 1$, and let $\{f_n\}: E \to R $ be a
sequence of functions which is equi-integrable. Show that if  $\lim_{n \to \infty} f_n(x) = f(x)$ a.e. $x$, then 
$$ \lim_{n \to \infty} \int_E |f_n(x) - f(x)| \, dm \, =\, 0.$$
\underbar{Hint}. Use Egorov's Theorem as in the bounded convergence theorem.

\bigskip

{\bf III.\,} In Chapter 2 we first prove the {Bounded Convergence Theorem} (using Egorov Theorem).
Then, we proved Fatou's Lemma (using the BCT) and
deduced from it the Monotone Convergence Theorem. Finally we proved the Dominated 
Convergence Theorem (using both BCT and MCT). Here we would like to prove these sequence of results in a different order.
Namely, prove:

{\bf a)}\,  Prove Fatou's Lemma {\it from} the MCT by showing
that for any sequence of measurable functions $\{f_n\}_{n \geq 1}$, 
$$  \int  \liminf_{ n\to \infty}  f_n  \, dm \,  \leq  \,  \liminf_{ n\to \infty} \int f_n \, dm . $$

\underbar{Hint}. \, Note that   $\inf_{ n \geq k} \,  f_n \leq f_j$ for any $ j \geq k$, whence  $\int \inf_{ n \geq k} \, f_n \, dm\,  \leq \, \inf_{j \geq k} \int  f_j.$ 
\medskip

{\bf b)}  Now prove the DCT  from Fatou's Lemma. 

\smallskip

\underbar{Hint}. Apply Fatou's Lemma to the nonnegative functions $g + f_n$ and  $g - f_n$.

\bigskip

{\bf IV.\,}   Consider the sequence of functions $f_n(x): = \frac{n}{1 +  (n x)^2}$.  For $a \in \Bbb R$ be a fixed number consider the Lebesgue integral $I_a(f_n)(x) := \int_a^\infty \, f_n(x)  \, dm$. Compute $\lim_{n \to \infty} I_a(f_n)(x)$ in each case: i) $a =0$  ii) $a > 0$  and iii) $a <0$. Carefully justify your calculations (recall the transformation of integrals under dilations). 

\bigskip 

{\bf V.\,}   Use the DCT to prove the following:   let $\{f_n\}_{n \ge 1}$ be a sequence of integrable functions on $\Bbb R^d$ such that $\sum_{n=1}^{\infty}\,  \int  |f_n(x)| \, dm < \infty$.  \, Show that   $\sum_{n=1}^{\infty}\,  f_n(x) \,$ {\it converges} a.e. \, $x \in \Bbb R^d$ to an {\it integrable function} and that  $\sum_{n=1}^{\infty}\,  \int  f_n(x) \, dm \, = \,  \int \, \sum_{n=1}^{\infty}\,  f_n(x) dm $.





\vskip .2in

\head {\underbar{SET 5:  Due Date 11/03/16}  }\endhead

\smallskip

\noindent {\bf Chapter 2 (pp 89-97) (Problems on Fubini-Tonelli)}:   4, 18, 19.


\vskip .15in
\underbar{Additional Problems:}
\medskip


{\bf I.\,}   Consider the function on $\Bbb R \times \Bbb R$ given by 
$$f(x, y) =  y e^{-(x^2 +1) y^2}\quad \text{ if }   x \geq 0, y \geq 0 \quad \text{and} \quad  0 \quad \text{ otherwise }.$$

a) Integrate $f(x,y)$ over $\Bbb R \times \Bbb R$   (justify your steps carefully)

b) Use a) to prove that $\int_0^\infty  e^{-x^2} \, d x \, =\, \frac{\sqrt{\pi}}{2}$.

c) Use b) and dilation to prove that $\int_{-\infty}^\infty  e^{- a x^2} \, d x \, =\, \sqrt{\frac{\pi}{a}}$.

\bigskip

{\bf II.\,}  Let $f(x)$ be a measurable function over $\Bbb R^{d_1}$ and $g(y)$  be a measurable function over $\Bbb R^{d_2}$. Prove that $F(x,y) = f(x) g(y) $ is a measurable function over $\Bbb R^{d_1 + d_2}$.


\vskip .15in

\head {\underbar{SET 6 - Due 12/01/16}  }\endhead
\vskip .1in

\noindent {\bf From Chapter  2 (pp 88-97)}:   Read Cor. 3.7- 3.8,  then do 7.  Recall Invariance (p. 73). Read Prop 3.9 then do 21a)b)c). 
 
\vskip .1in
\underbar{Additional Problems}
\bigskip
{\bf I.\,} Let $s$ be a fixed positive number.  Prove that
$$
\int_0^\infty e^{-s x} \, \frac{\sin^{2}x}{x} \, dx 
= \frac{1}{4} \log(1 + 4 s^{-2}) $$
by integrating $e^{-s x}\sin(2xy)$ with respect to $x \in (0,\infty)$,
$y \in (0,1)$ and with respect to $y \in (0,1)$, $x \in (0,\infty)$.  Justify all your steps.
\ \ \ ({\bf Hints.}  $\cos(2 \theta) = 1 - 2 \sin^2\theta$.  In order to do one of the integrations,
either integrate by parts twice or use the definition of the appropriate trigonometric function in 
terms of complex exponentials.)


\bigskip
{\bf II.\,}   Consider the function $f(x,y):=  e^{-xy} - 2 e^{-2xy}$ where $ x \in [ 0, \infty)$ and $ y \in  [ 0,1]$. 

i)  Prove that for a.e.  $y \in [0,1]$  $f^{y}$ is integrable on $[0, \infty)$ with respect to  $m_{\Bbb R}$. 

ii) Prove that for a.e.  $x \in [0,\infty)$  $f^{x}$ is integrable on $[0, 1]$ with respect to  $m_{\Bbb R}$.

iii) Use Fubini to prove that $f(x,y)$ is not integrable
on $[ 0, \infty) \times [ 0,1]$ with respect to  $m_{\Bbb R^2}$.   

\bigskip
\noindent {\bf From Chapter  3 (pp 145-146 -Section 5)}: \, 1  (part c) is involved) , 4, 5. 

\bigskip
\noindent {\bf From Chapter  3 (pp 152- Section 6) Bonus Problem}: \,  1  \,(this problem is based on a good understanding of Lemma 1.2  p.102 and of Lemma 3.9 p.128-definition of Vitali covering is just before the statement of Lemma 3.9.)




\vskip .15in


\head {\underbar{SET 8 - Due 12/08/16} }\endhead

\vskip .1in

 

\noindent {\bf  Chapter  3 (pp 145-150- Section 5)}:  \,  7, 11, 12, 15, 16,  22. 


\vskip .1in

{\bf Note}:   Please start working on the above. I'll add a couple more on BV and absolute continuity. 



\vskip .2in


\enddocument


                                          




%%%%%%%%%%%%%%%%%%%%%%%%%%%%%%%%%%%%%%%%%%%%%%%%%%
%%%%%%%%%%%%%%%%%%%%%%%%%%%%%%%%%%%%%%%%%%%%%%%%%%
%%%%%%%%%%%%%%%%%%%%%%%%%%%%%%%%%%%%%%%%%%%%%%%%%%%

\medskip


{\bf I.}  First recall that a set $E$ in $\Bbb R^d$ is closed if and only if $E$ contains all its limit points; in particular any convergent sequence in $E$ has limit in $E$.
Next recall that a set $K \subset \Bbb R^d$ is said to be compact if $K$ is closed and bounded. 

\medskip

Show that the following are equivalent:

\smallskip

i)  \, $K$ is compact

\smallskip

ii) \,  Any cover of $K$ by open sets, $K \subset \cup_{\alpha} \Cal{O}_{\alpha}, \quad \Cal{O}_{\alpha}$ open,  contains a finite sub-cover $$K \subset \cup_{j=1}^M  \Cal{O}_{j}, \qquad \text{ for some } \, \, M \geq 1$$ 

\smallskip

iii)\, Any sequence $\{y_n\}_{n \geq 1} \subset K$, contains a convergent subsequence whose limit is in $K$.


\underbar{Additional Problem:}    Suppose that A is a measurable set in $\Bbb R^d$ with $m(A) > 0$. Show that for any
$q < m(A)$ there exist a measurable set $B\subset  A$ with $m(B) = q$. 

(Hint: Prove it first for the case that $m(A)=p < \infty$. Use then the intermediate value theorem for
$A \cap B_R(0)$.)







