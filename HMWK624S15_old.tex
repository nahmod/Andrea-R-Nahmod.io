\magnification=1200
\input amstex

\documentstyle{amsppt}
\NoBlackBoxes
%\NoPageNumbers
\pagewidth{16.5truecm}
\pageheight{22truecm}

{\catcode`@=11
\gdef\nologo{\let\logo@\empty}
\catcode`@=12}
\nologo
%\hcorrection{.3in}

\centerline{ \bf  M624 HOMEWORK  -- SPRING 2015}
\vskip .1in
\centerline{ Prof. Andrea R. Nahmod }
\vskip .2in

\document

\head {\underbar{SET 1 - Due 02/05/15 }  }\endhead
\vskip .2in

%Last semester you did the following problems already:  4, 5 (Ch.  3, pp 146-Section 5) and  1 (Ch.  3, pp 152- Section 6) which is based on a good understanding of Lemma 1.2  p.102 and of Lemma 3.9 p.128-definition of Vitali covering is just before the statement of Lemma 3.9.  Now do:

%\smallskip

\noindent {\bf From Chapter 3 (pp 145-152)}:  11, 12, 14, 15, \, 16a)b), 19,  23,  32. 
\smallskip

\noindent For 15, write each increasing fc. as a continuous one plus its associated Jump fc.

\medskip

\head {\underbar{SET 2 - Due 02/19/15 }  }\endhead

\vskip .2in

\noindent {\bf From Chapter 3 (pp 153)}:  4.

\vskip .2in 

\noindent {\bf From Chapter 4:}

Read (again!) carefully Theorem 2.2 (Riesz-Fisher) on Chaper 2 (p. 70) and compare with Theorem 1.2 Chapter 4 (p. 159). Fill in the gaps (the {\it why?} in class) in the proof of Th. 1.2 Ch 4) and then do:  

\smallskip


{\bf Pb. I.}  For any $1 \leq p < \infty$ consider the space
$$L^p(\Bbb R^d):= \{ f: \Bbb R^d \to \Bbb C, \text{ measurable}, \, : \, \| f\|_{L^p(\Bbb R^d)}:= \left( \int_{\Bbb R^d}  | f(x) |^p \, dm  \right)^{\frac{1}{p}} < \infty \}.$$
{\underbar{Assume}} that $ \| f\|_{L^p(\Bbb R^d)}$ is a norm ( {\it challenge:} can you guess what would you need to prove the triangle inequality when $p \neq 2$?), whence $d_p(f,g) : =  \| f - g\|_{L^p(\Bbb R^d)}$ defines a metric and $L^p$ is a metric space.  {\bf Prove} that $L^p(\Bbb R^d)$  is {\it complete}.

\bigskip

\noindent {\bf From Chapter 4 (pp 193-194)}:  1, 2, 3, 4, 5, 6, 7, 8a).

\bigskip

\noindent {\bf From Chapter 4 (pp 202)}:   $2^{\ast}$a)b)

\head {\underbar{SET 3 - Due 02/26/15 }  }\endhead
 
\noindent {\bf From Chapter 4 (pp 195-197)}:  10, 11, 12, 13, 20


\medskip

{\bf Pb. II.} Consider the subspace $\Cal S$ of $L^2([0,1])$ spanned by the functions: $1 ,\,  x,$ and  $x^3$. 

a) Find an orthonormal basis of $\Cal S$.

b)  Let $P_{\Cal S}$  denote the orthogonal projection on the subspace $\Cal S$, compute $P_{\Cal S} x^2$.
  
\medskip

{\bf Pb. III.}  Consider  a function $ f \in L^2([-\pi, \pi])$ whose Fourier series is $\sum_{n \in \Bbb Z} a_n e^{inx} = \lim_{N \to \infty} S_N(f)(x)$ - equal a.e. to $f(x)$. 
Show that on any subinterval $[a, b] \subset [-\pi, \pi] $ we have,
$$ \int_a^b f(x)\, dx \, = \, \sum_n   \int_a^b  \, a_n \, e^{i n x} \, dx. $$ In particular if $g(x)= \int_a^x f(y) dy $, the Fourier coefficients and series of $g(x)$ 
can be obtained from $a_n$, the Fourier  coefficients of $f$.

\medskip

{\bf Pb. IV.} For $0 < \alpha < 1$, we say that  a function $f$ is  $C^{\alpha}$-H\"older
continuous with exponent  $\alpha$ if there exists a constant $c=c_{\alpha}>0$ such that $|f(x)- f(y)| \leq c\, | x -y|^{\alpha} $ for all $x, y$. 
For $k \in \Bbb N$, we can also define the space $C^{k, \alpha}$ to be that of functions which are $k$-th times differentiable and whose $k$-th derivative is
$C^{\alpha}$-H\"older continuous (we could relabel $C^{\alpha}$ as $C^{0,\alpha}$). 

Consider now $f$ a $2\pi$-periodic $C^{k, \alpha}$ function. If $a_n$ are the Fourier coefficients of $f$,  show that for some $C>0$ independent of $n$, 
$$ |a_n| \leq \frac{C}{|n|^{k+\alpha} } $$

\bigskip

\head {\underbar{SET 4 - Due 03/05/15 }  }\endhead

\noindent {\bf From Chapter 4 (pp 197-202)}:  18, 19, 21a), 22, 24, 26.


\bigskip

\underbar{Additional Problems ({\bf do} but do not turn in)}:  29 (p 199-200) and $6^{\ast}$ (p. 203-204). These are about Fredholm's Alternative for compact operators.

\bigskip

\head {\underbar{SET 5 - Due 03/12/15 }  }\endhead

\noindent {\bf From Chapter 4 (pp 196-202)}:  15, 23, 25, 28, 30, 32, 33.

\noindent {\bf From Chapter 6 (pp  317-322)}:  1, 2a), 3 

\bigskip

\head {\underbar{SET 6 - Due 04/02/15 }  }\endhead


\noindent {\bf From Chapter 6 (pp 317-322)}:  5,  8, 10, 11a)b), 16a)b)

\medskip

\underbar{Additional Problems:}  

\smallskip

\noindent {\bf(A1)} \, Let $\nu$ be a signed measure on $(X, \Cal M)$. Show that for any $E \in \Cal M$

$$ \align |\nu|(E) &= \\
&= \sup \{ \sum_{k=1}^K\,  |\nu(E_k)| \, : \,  E_1, \dots E_K  \text{ are disjoint and }\,  E=\cup_{k=1}^K E_k  \, \} \tag{1} \\
&=   \sup \{ \, \sum_{k=1}^{\infty}\,  |\nu(E_k)| \, : \,  E_1,  E_2,  \dots   \text{ are disjoint and } \,  E=\cup_{k=1}^{\infty} E_k \, \}  \tag{2} \\ 
&=   \sup \{ | \int_{E}   f  d \nu |  \, : \,  |f|  \leq 1 \}  \tag{3}
\endalign $$ 

You may want to proceed for example by proving that  $(1) \leq (2) \leq (3) \leq (1)$. 

\medskip

\noindent {\bf(A2)} \, Let $F \in BV([a,b])$ and right continuous. Let $G(x)= |\mu_F|([a, x])$. Show that $|\mu_F|=\mu_{T_F}$ by showing that $G=T_F$. To do so you may proceed by proving:

1) \,  $T_F  \leq G$  (use definition of $T_F$).

2) \, $|\mu_F(E)| \leq \mu_{T_F} (E) $ for any Borel set $E$  (do for an interval first).

3)  Show that  $|\mu_F| \leq \mu_{T_F}$ and hence $G \leq T_F$  (use (A1)). 

\medskip

\noindent {\bf(A3)}  Let $F$ and $G$ be $BV([a,b])$ and right continuous. Let $\mu_F$ and $\mu_G$ be the corresponding signed Borel measures
(recall these measures are uniquely determined by -say-  $\mu_F( c, d])= F(d) - F(c)$).  
\smallskip

a)  Show that if either $F$ or $G$ are continuous the following {\it integration by parts} formula holds:
$$ \int_{(a, b]} F d\mu_G + \int_{(a, b]} G d\mu_F \, =\, F(b)G(b) - F(a)G(a) $$

\smallskip
b)  If  $F$ and $G$ are absolutely continuous then 
$$ \int_{(a, b]} F G^{\prime} \, dx \,+\,  \int_{(a, b]} G F^{\prime} \, dx \, =\, F(b)G(b) - F(a)G(a) $$

\bigskip

\underbar{Problems {\bf to do} (but do not turn in)}:  \, 14, 16c)d)e)f) 
 
 \bigskip
 
 \bigskip
 
 \head {\underbar{SET 7 - Due 04/16/15 }  }\endhead
\medskip

\noindent {\bf From Chapter 1 of [SS, Vol. 4] (pp 34-43)}:  1, 3, 5, 6, 7, 8.


\head {\underbar{SET 8 - Due 04/30/15}}\endhead
\medskip

\noindent {\bf From Chapter 1 of [SS, Vol. 4] (pp 36-43)}: 9,  12 (do this on $\Bbb R^n$ with Lebesgue measure), 13, 15, 16, 17, 19, 20, 34, 35.



 
 \enddocument



                                               













