\documentclass[12pt]{article}
\usepackage{times}
\usepackage{amssymb,amsmath,amsthm}
\usepackage{amsfonts}
\newcommand{\limj}{\lim_{j\rightarrow\infty}}
\newcommand{\limn}{\lim_{n\rightarrow\infty}}
\newcommand{\limx}{\lim_{x\rightarrow\infty}}
\newcommand{\noi}{\noindent}
\newcommand{\R}{I\!\!R}
\newcommand{\Rd}{\R^d}
\newcommand{\rd}{\R^d}
\newcommand{\ZZ}{{Z\!\!\!Z}}
\newcommand{\X}{{\cal X}}
\newcommand{\M}{{\cal M}}
\newcommand{\N}{{I\!\!N}}
\newcommand{\C}{l\!\!\!C}
\newcommand{\T}{I\!\|\|T}
\newcommand{\ds}{\displaystyle}
\newcommand{\ve}{\varepsilon}
\newcommand{\lan}{\langle}
\newcommand{\ran}{\rangle}
\newcommand{\goto}{\rightarrow}
\newcommand{\hf}{\widehat f}
\setlength{\textheight}{22.5cm}
\setlength{\textwidth}{16cm}
\setlength{\topmargin}{-1cm}
\setlength{\oddsidemargin}{0cm}
\setlength{\evensidemargin}{0cm}

\begin{document}

{\bf \Large NAME:}
\vspace{.2in}

\begin{center}
{\bf MATH  623  \quad FINAL EXAM}\\
\underbar{\bf Due}: Friday, December 15th, 2017 no later than 11AM
\end{center}
\vskip .5in

{\bf Instructions} 
\begin{enumerate}

\item This exam consists of four (4) problems all counted equally 
for a total of $100\%$.
\vskip .2in 

\item You may consult Stein-Shakarchi's Vol. III and class material (class notes, homeworks done, handouts) \underbar{\bf only}.
No other books or notes are permitted. 

\vskip .2in 

\item You should work on the problems alone; do not discuss the problems with other people or classmates. 
You may ask me any questions you have.

\vskip .2in 
\item {\bf\underbar{Type}} each problem and its solution in an ordered fashion (new page for each problem)  and staple them all together with this cover. 
Insert additional pages if needed.

\vskip .2in 

\item State explicitly all results that you use in your proofs and
verify that these results apply.
\vskip .2in 



\item Show all your work and \underbar{ justify } the steps in your proofs. 





\end{enumerate}
\vspace{.25in}

{\bf Conventions}
\begin{enumerate}
%
%\item For a set $A$, $1_A$ denotes the
%indicator function or characteristic function of $A$. 
%
%\vskip .1in 
%
\item If a measure is not specified, use Lebesgue measure. This measure is denoted by $dm$ or $dx$.
%
%\vskip .1in 
%
%\item If a $\sigma$-algebra on $\R$ is not specified, use the
%Borel $\sigma$-algebra ${\cal B}(\R)$
\end{enumerate}
%


\begin{enumerate}
\newpage
\item
\vspace{.2in}
% Problem 1


a) Let $F: [-1, 4] \to \Bbb R$ be the function defined by  $F(x)= 2|x| - |x-2| $. 
Prove that $F$ is of bounded variation and compute the total variation  $T_F(-1, 4)$ of $F$ over $[-1,4]$. 

\vskip .5in


%b) Prove that if $F \in C^1([a, b])$  then $F$ is $BV([a, b])$. Recall $C^1$ means  continuous on $[a,b]$, differentiable on $(a, b)$ with continuous derivative. 


b) Let $F \in BV([a, b])$ and let  $c$ be such that $ a <c <b$. Show that 
$$ T_F(a, b) \, =\, T_F(a, c) \, +\, T_F(c, b) $$

\vskip .5in

c) \, Show that if $L$ is Lipschitz continuous and $F$ is of bounded variation then the composition $L \circ F$ is of 
bounded variation.  Recall $L$ Lipschitz continuous means that there exists $M>0$ such that 
$|L(x) - L(y)| \le M |x-y|$ for all $x,y$.

%\newpage
%\item
%\vspace{.2in}
%% Problem 2
%
%
%\vskip 1in
%
%b) \, Suppose that $F$ is of bounded variation and $F(x) > \eta >0$ for all  $x \in [a, b]$ ($F(x)$ is uniformly strictly positive) . Show that there exist two monotone increasing functions $G$ and $H$ such that for all $x \in [a, b]$ we have that 
%$$F(x) = \frac{G(x)}{H(x)}$$ 
%\underbar{Hint.}  \, Use part a) with a suitable function $L$.
%

\newpage
\item
\vspace{.2in}
% Problem  5
 \, For each fixed $N\geq 1$ natural number, let $\mathcal R_N :
= [0, N] \times [0, \infty)$, be a region in  $[0, \infty) \times [0,
\infty)$ endowed with the Lebesgue measure.


(a) Consider the continuous function
$f(x, t) := (\text{sin} \,x) e^{-x t}$  defined
on  $[0, \infty) \times [0, \infty)$. 
Prove that for each fixed $N$,  $f$ is integrable over $\mathcal R_N$; i.e.,
$$ \int \! \! \int_{\mathcal R_N} \, \,  | f(x, t) | \, dx dt \, < \infty. $$ 


(Hint. Use Tonelli's Theorem)


\vskip .2in 


(b)  Use part (a) and
the fact that for any $x \in \mathbb R$ , $x \neq 0$ $$ \frac{1}{x}
= \int_0^{\infty}  \, e^{-x t } \, dt $$  
to rigorously prove that   
$$ \lim_{N \to \infty} \, \int_0^{N} \, \frac{\sin \,x }{x} \, dx \, =
\, \frac{\pi}{2}.$$


(Hints. Use Fubini's Theorem in conjunction with the Dominated 
Convergence Theorem.)





\newpage 
%
%
\item
\vspace{.2in}
%% Problem  3
 Let $f \in L^1(\mathbb R)$ and let ${\widehat f}$ be its Fourier transform
   defined by $$\widehat f (\xi) := \, \int_{\mathbb R} f(x) \, e^{- i \xi x} \, dx
   $$ for all $\xi \in \mathbb R$. 
\vskip .2in 

a) Prove that $\hf$ is uniformly continuous on $\mathbb R$.
\underbar{Hint}. Prove and use the fact that for real numbers $a, b$ and $z$, \, \, $| e^{i a z} - e^{i b z}| \leq |z| |a-b|$.

\vskip .5in

b)  Prove that $\hf$ is also bounded on $\mathbb R$.  What is the $\sup_{ \xi \in \mathbb R} | \hf(\xi) |$ less than or equal to?


\vskip .5in


c) Assume  $f \in L^1(\mathbb R)$ satisfies
\begin{eqnarray*}  &\int_{|x| \le N}  \, |x| \, | f (x) | \, dx \leq N^{1/2} \\ \\
&\int_{ |x| > N } | f(x)| \, dx  \leq \dfrac{1}{N^{1/2}} \end{eqnarray*}
for all $0 < N < \infty$. \, Show then that there exists a constant $C>0$ and $0< \beta <1$ such that 
for all $\xi$ and $\eta \in \mathbb R$, 
 $$ | {\widehat f}(\xi) - {\widehat f}(\eta) | \le C \, | \xi -
\eta |^{\beta} $$ 
( in other words, $\widehat f$ is H\"older continuous of order $\beta$). 

\vskip .1in

\underbar{Hint}. Obtain an estimate for 
$\int_{ |x| \le N} ( e^{-i x \xi} - e^{-i
x \eta} ) \, f(x) \, dx $  and another estimate for 
$\int_{ |x| > N} ( e^{-i x \xi} - e^{-i x \eta} ) \, f(x) \, dx $. 
Then optimize your choice of $N$ (in terms of $|\xi- \eta|$) to obtain the desired estimate. 

\vskip .5in

d) Now assume that in addition to having $f\in L^1(\mathbb R)$, the
function $x f(x)$ is also integrable; that is,  
$\int_{\mathbb R} \, | x f(x) | \, dx < \infty$. 
\vskip .1in
Show then that:   $\hf$ is  differentiable  and moreover that, 
$$\frac{d}{d\xi} \hf(\xi) = {\widehat {( -i x f)} } (\xi). $$ 

\underbar{Hint}. Use the Dominated Convergence Theorem. 

\newpage 

\item
\vspace{.2in}
% Problem  4
 Let $f \in L^1(\mathbb R)$ and $r >0$. Set $$ A_r(f)(x) :=
   \frac{1}{2r} \int_{x-r}^{x+r} \, f(y) \, dy .$$

\vskip .1in 

(a) Show that the average $A_r(f)(x)$ is a continuous function in {\bf both}  $x$ and $r$.

\vskip 1in 

(b) Show that if in addition $f$ is continuous, then $$\lim_{r \to 0}
A_r(f)(x) \, = \, f(x) .$$
\vskip 1in

(c) Show that $A_r$ is a contraction in $L^1(\mathbb R)$ in the sense that
$$\Vert A_r(f) \Vert_{L^1(\mathbb R)}  \leq \Vert f \Vert_{L^1(\mathbb R)} .$$
\underbar{Hint.} Use Tonelli)
\vskip 1in 

(d) Use (b) and (c) to show that if $f \in L^1(\mathbb R)$ then 
$$ \lim_{r \to 0} \Vert A_r(f) - f \Vert_{L^1(\mathbb R)} \, = \, 0 . $$

\underbar{Hint}. You may use without proof the fact that $L^1(\mathbb R)$ functions can be approximated in
$L^1(\mathbb R)$ by continuous functions. 




\end{enumerate} 
\end{document}
                                   