\input amstex
\documentstyle{amsppt}
\nologo
\magnification=\magstephalf
\pagewidth{6.5truein}
\pageheight{9truein}
\parskip=5pt
\abovedisplayskip=11pt          %plus 3pt minus 3pt
\belowdisplayskip=11pt          %plus 3pt minus 3ptf

\NoBlackBoxes
\def\card{\operatorname{card}}
\def\intinfin{\int_{-\infty}^{\infty}}
\def\const{\operatorname{const.}}
\def\dist{\operatorname{dist}}
\def\con{\operatorname{const}}
\def\sgn{\operatorname{sgn}}
\def\supp{\operatorname{supp}}
\def\sign{{\rm sgn}}
\def\conphi{{\text const}_{\phi}}
\def\conpsi{{\text const}_{\psi}}
\def\J{{\cal J}}
\def\Q{{\cal Q}}
\def\A{{\Cal A}}
\def\B{{\Cal B}}
\def\C{{\Cal C}}
\def\D{{\Cal D}}
\def\E{{\Cal E}}
\def\F{{\Cal F}}
\def\G{{\Cal G}}
\def\H{{\Cal H}}
\def\I{{\Cal I}}
\def\J{{\Cal J}}
\def\K{{\Cal K}}
\def\L{{\Cal L}}
\def\J{{\Cal J}}
\def\M{{\Cal M}}
\def\Q{{\Cal Q}}
\def\P{{\Cal P}}
\def\S{{\Cal S}}
\def\R{{\Cal R}}
\def\T{{\Cal T}}
\def\V{{\Cal V}}
\def\W{{\Cal W}}
\def\gA{{\frak A}}
\def\gB{{\frak B}}
\def\gF{{\frak F}}
\def\gG{{\frak G}}
\def\gH{{\frak H}}
\def\gN{{\frak N}}
\def\gP{{\frak P}}
\def\gT{{\frak T}}
\def\IC{{\Bbb C}}
\def\II{{\Bbb I}}
\def\IJ{{\Bbb J}}
\def\IN{{\Bbb N}}
\def\IP{{\Bbb P}}
\def\IQ{{\Bbb Q}}
\def\nat{{\Bbb N}}
\def\que{{\Bbb Q}}
\def\IR{{\Bbb R}}
\def\real{{\Bbb R}}
\def\IS{{\Bbb S}}
\def\IT{{\Bbb T}}
\def\IW{{\Bbb W}}
\def\IZ{{\Bbb Z}}
\def\zed{{\Bbb Z}}
%\def\wal#1{\omega_{#1}}
%\def\stof#1{\text{\rm w}_{#1}}
\def\lto{\longrightarrow}
\def\conj{\overline}
\def\sign{\text{sign}}
\def\myskip{\noalign{\vskip6pt}}
%\def\bmatrix#1{\left[ \matrix #1\endmatrix \right]}
\def\varep{\varepsilon}
%\def\ch{\raise 0.3ex\hbox{$\chi$}\kern-.15em}
\def\iprec{\mathop{\prec}_{i}}
\def\3prec{\mathop{\prec}_{3}}
\def\liirr{{L^2_{\rho_r}}}
\def\so{\text{\rm SO}}
\def\bR{\text{\bf R}}
%\def\wal#1{\omega_{#1}}
%\def\stof#1{{\rm w}_{#1}}
%\def\Lip{{\rm Lip}}
\overfullrule=0pt
%\def\ch{\raise 0.3ex\hbox{$\chi$}\kern-.15em}
%\def\real{{\Bbb R}}

\def\dist{\operatorname{dist}}
\def\IH{{\Bbb H}}
\def\gF{{\frak F}}
\def\gN{{\frak N}}
\def\IS{{\Bbb S}}

\topmatter
\title Problem Set  3 \endtitle
\date Due TUESDAY October 23rd,  2001 \enddate
\endtopmatter

\document

\subhead{ Problems from Folland }\endsubhead :  (pg. 27) 6, 11, 13, 14  and (pg 32) 17, 18, 20, 22a) 

\enddocument 





\proclaim{Problem 1 ($\#2$ from Folland p.24)} Complete the reminding cases of the Proof of Proposition 1.2 ( Folland p. 22) by proving 
that ${\Cal B}_{\Bbb R} \subseteq {\Cal M}( {\Cal E}_j ) $ for all $j \ge 3$. 

\endproclaim


\proclaim{Problem 2 ($\# 4 $ from Folland p.24)} An algebra $\Cal A$ is a $\sigma$-algebra if and only if $\Cal A$ is closed under countable increasing unions
(by the latter it is meant that if $\{E_j\}_{j=1}^{\infty} \subset {\Cal A}$ and 
$E_1 \subset E_2 \subset E_3 \subset \cdots $, then $\bigcup_{j=1}^{\infty} E_j \in {\Cal A} $ ).  


\endproclaim 

\proclaim{Problem 3 ($\# 5 $ from Folland p.24) }  If $ {\Cal M}$ is the $\sigma$-algebra generated by $\Cal E$, then $\Cal M$ is the union of the $\sigma$-algebras generated by $\Cal F$ as $\Cal F$ ranges over all countable subsets of $\Cal E$. (Hint: show that the latter object is a $\sigma$-algebra). 

\endproclaim



\proclaim{Problem 4  ($\# 7 $ from Folland p.27) } If $\mu_1, \mu_2, \dots, \mu_n$ are measures on $(X, \Cal M)$ and $ a_1, a_2, \dots, a_n \in [0, \infty)$, then $\sum_{j=1}^n a_j \mu_j$ is a measure on $(X, \Cal M)$. 

\endproclaim


\proclaim{Problem 5 ($\# 9 $ from Folland p.27) } If $(X, \Cal M, \mu)$ is a measure space and $E, F \in \Cal M$, then $ \mu(E) + \mu(F) = \mu(E \cup F) + \mu(E \cap F)$


\endproclaim



\proclaim{Problem 6 ($\# 10 $ from Folland p.27) } Given a measure space $(X, \Cal M, \mu)$ abd $E \in \Cal M$, define $$\mu_E(A) = \mu (A \cap E) \qquad \text{ for } A \in \Cal M. $$ Prove that $\mu_E$ is a measure. 


\endproclaim


\proclaim{Problem 7 } Provide a complete proof of Proposition 1.4 p. 23 of Folland. That is prove the following:

Suppose that ${\Cal M}_{\alpha}$ is generated by ${\Cal E}_{\alpha}$, $\alpha \in A$. Then $\bigoplus_{\alpha \in A} {\Cal M}_{\alpha}$ is generated by $$\F_1 := \{ \pi^{-1}_{\alpha} (E_{\alpha}) \, : \, E_{\alpha} \in {\Cal E}_{\alpha}, \alpha \in A \}. $$
If $A$ is countable and $X_{\alpha} \in {\Cal E}_{\alpha}$ for all $\alpha$, $\bigoplus_{\alpha \in A} {\Cal M}_{\alpha}$ is generated by 
$$\F_2 := \{ \Pi_{\alpha \in A} (E_{\alpha}) \, : \, E_{\alpha} \in {\Cal E}_{\alpha} \}. $$
\endproclaim 
{\bf Note} : Verbatim reproduction of the proof in book with no further explanations or justifications of claims is  not acceptable. If you use a previous result please state \underbar{clearly} what are you using instead of just refering to numbering in the book. 

 


\enddocument























