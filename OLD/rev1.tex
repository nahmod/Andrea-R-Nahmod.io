\input amstex
\document 

\bigskip

Dear All: 

\vskip .1in

You have already had lots (!) of practice by doing your homework and, by now, having 
reviewed all class material including the many examples we did in class {\it in addition} to 
those already in the book (I purposedly do different ones). All these examples already conform a pretty good list of problems you should try to re do {\bf first and on your own} as {\it good practice problems}. Thus, by reviewing all of the above and provided you can do the problems on your own (!)  you should be well prepared for the exam. 

\vskip .1in 

{\bf Only after having done the above (!)}: should you wish to practice more, here are a list of 
some more and fresh problems to practice. These are by no means meant to be 
``questions like the ones in the exam'' careful with that kind of thinking....

\vskip .3in 

Pgs $356-358 :  18, 36, 40, 42, 44 , 64  $

\bigskip

Pgs $388-390: 3,4, 63$

\bigskip

Pgs $400:$  do variations of  $62, 63 $ ( see examples done in class)

\bigskip

Pgs. $407-408 :  12, 13, 14 54, 56 and 61$ ( in $61$ change to $C'(x) = 3 + x e^{-x/2}$)

\bigskip

Pgs. $417-418: 74, 76, 77, 78, 79, 80. $

\bigskip

Pgs. $426-428 :  2a) b) , 4 , 8c), 14 , 16, 18, 29, 35, 68, 70$
\vskip .1in 
Note for problem $2b)$ :
Change it so that it reads "express in terms of limits first and then
calculate by definition.
 
Also do something that asks the "viceversa"
question: i.e.  given the limit of a Riemann sum ask what
definite integral represents if- for example- the partition
increments $\Delta x_i$ are given but not the interval of integration. 
Example : express the following limit as a definite integral:

$$\lim_{n \to \infty} \sum_{i=1}^n \frac{e^{x_i}}{1 + x_i} \frac{4}{n}
\qquad \text{ where } x_i = 1 + \frac{4}{n} $$

\bigskip 

Pgs. $463-464:   2, 5, 6, 16a) 32a) $

\bigskip 

Page $467:  14 $











\enddocument 