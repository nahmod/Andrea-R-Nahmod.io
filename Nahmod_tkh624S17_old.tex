


\documentclass[12pt]{article}
\usepackage{times}
\usepackage{amssymb,amsmath,amsthm}
\usepackage{amsfonts}
\newcommand{\limj}{\lim_{j\rightarrow\infty}}
\newcommand{\limn}{\lim_{n\rightarrow\infty}}
\newcommand{\limx}{\lim_{x\rightarrow\infty}}
\newcommand{\noi}{\noindent}
\newcommand{\R}{I\!\!R}
\newcommand{\Rd}{\R^d}
\newcommand{\rd}{\R^d}
\newcommand{\ZZ}{{Z\!\!\!Z}}
\newcommand{\X}{{\cal X}}
\newcommand{\M}{{\cal M}}
\newcommand{\N}{{I\!\!N}}
\newcommand{\C}{l\!\!\!C}
\newcommand{\T}{I\!\|\|T}
\newcommand{\ds}{\displaystyle}
\newcommand{\ve}{\varepsilon}
\newcommand{\lan}{\langle}
\newcommand{\ran}{\rangle}
\newcommand{\goto}{\rightarrow}
\setlength{\textheight}{22.5cm}
\setlength{\textwidth}{16cm}
\setlength{\topmargin}{-1cm}
\setlength{\oddsidemargin}{0cm}
\setlength{\evensidemargin}{0cm}

\begin{document}

{\bf \Large NAME:}
\vspace{.2in}

\begin{center}
{\bf MATH  624  \quad Final Take Home Exam}\\
\underbar{\bf Due}: Wednesday, May 10th 2017 no later than 5:30PM
\end{center}
\vskip 1in

{\bf Instructions} 
\begin{enumerate}

\item This exam consists of five (5) problems all counted equally 
for a total of $100\%$.
\vskip .5in 

\item You may consult Stein-Shakarchi books III and IV, your homework or the class notes \underbar{\bf only}.
No other books or notes are permitted. 

\vskip .5in 

\item You should work on the problems alone; do not discuss the problems with other people or classmates. 
You may ask me any questions you have.
\vskip .5in 

\item State explicitly all results that you use in your proofs and
verify that these results apply.
\vskip .5in 

\item  Please \underbar{{\bf type}} your full work and answers \underbar{clearly} after each problem 
and attached each answer to the stated problem
\vskip .5in 

\item Show all your work and \underbar{ justify each and all steps} in your proofs. 





\end{enumerate}
%\vspace{.25in}

%{\bf Conventions}
%\begin{enumerate}
%
%%\item For a set $A$, $\chi_A$ denotes the
%%indicator function or characteristic function of $A$. 
%
%\vskip .1in 
%
%\item If a measure is not specified, use Lebesgue measure on
%$\mathbb R$. This measure is denoted by $m$. 
%
%\vskip .1in 
%
%\item If a $\sigma$-algebra on $\mathbb R^d$ is not specified, use the
%Borel $\sigma$-algebra ${\cal B}(\mathbb R^d)$
%\end{enumerate}
%


\begin{enumerate}
\newpage
\item
\vspace{.2in}
% Problem 1

 Let $\mu$ and $\nu$ be two positive measures on a 
measurable space $(X, \mathcal M)$. Assume that $\nu$ is finite. Show that that following are
equivalent:


\vskip .2in 


a)   $\nu \, \ll \, \mu$  holds. 


\vskip .2in


b) For each $\{A_n \}_{n \ge 1}$ in $\mathcal M$ with $\lim_{n
\to \infty} \mu(A_n) = 0$,  we have that  $\lim_{n \to \infty}
\nu(A_n) = 0$. 


\vskip .2in


(c) For every $\varepsilon >0$ there exists $\delta =
\delta(\varepsilon) >0$ such that whenever $A \in \mathcal M$ satisfies $
\mu(A) < \delta$, then $ \nu(A) < \varepsilon$ holds.  



\newpage


\item
\vspace{.2in}
% Problem  3

a) Let $(X, \|\cdot\|)$ be a normed vector space. Show that the following are equivalent:

i) $X$ is complete (hence $X$ is Banach).
\smallskip

ii)  If  $\{x_n\}_{n\ge 1} \subseteq X$ satisfies that $\sum_{n=1}^{\infty} \|x_n\| < \infty$ then $\sum_{n=1}^{\infty} x_n$  \underbar {converges in} $X$

\medskip

b)  Let $(X, \mathcal M)$ be a measurable space. Show that $M(X):=$ the space of all signed {\bf finite} measures on $(X, \mathcal M)$ together with the norm $$\| \mu\|= |\mu|(X) $$ is a Banach space. 
\smallskip

Here, $|\mu|$ denotes the total variation of $\mu$. You may assume without proof that $M(X)$ is a vector space over $\mathbb R$ and that the total variation is a norm.

\vskip .2in

\underbar{Hints.}  Use part a) to prove completeness by establishing ii). 

To prove ii) suppose $\sum_n \| \mu_n\|  < \infty$ and consider -for example- $\nu:= \sum_{n=1}^{\infty}  |\mu_n|$,  a positive finite measure (why?).  {\bf Prove} that $ \mu_n$ are all absolutely continuous w.r.t. $\nu$. Then use the Radon-Nikodym theorem to find $f_n \in L^1(d\nu)$ (here recall $\mu_n$ are signed {\bf finite}). Use this to find an $f \in L^1(d\nu)$ and then a $\mu \in  M(X)$ such that $\| \mu - \sum_{n=1}^N \mu_n \| \to 0$ as $N \to \infty$. 


\newpage

\item
\vspace{.2in}
% Problem  4
Let $X$ be a Banach space and $X^{\ast}$ its dual.  Recall a sequence $\{x_n\}_{n\geq 1}$ is said to {\it converge weakly} to $x$ if 
$$\lim_{n\to \infty} \ell(x_n) \,=\, \ell(x)$$ for any $\ell \in  X^{\ast}$.

\begin{enumerate}
\item Show that convergence implies weak convergence.

\item Show that if $X=H$ a separable Hilbert space and $\{x_n\}_n$ is an orthonormal  basis of $H$ then $x_n$ converges weakly to $0$
but it does {\bf not} converge strongly.

\smallskip

\item Suppose $X= \ell^1(\mathbb N)$. Show that if $x_n$ converges weakly in $\ell^1$ then it converges in $\ell^1$ (use what's the dual of $\ell^1$). 

\item Let  $f_n := n \, 1_{(0, \frac{1}{n})} \in L^p$ (for any $p \geq 1$). Show that $f_n$ converges to $0$ in measure and a.e. but it does not converge to $0$ weakly in $L^p$ for any $p$. 

\end{enumerate}
\newpage

\item 

 \, Let $f(x) = |x|$, $x \in \mathbb R$. Let $\mathcal M$ be the {\it smallest} $\sigma$-algebra with respect to 
which $f$ is measurable. 
\vskip .2in

(a) Characterize $\mathcal M$, i.e. describe the measurable sets. Characterize also 
the the measurable functions with respect to $\mathcal M$.  
\vskip .1in 

Hint: For a given $J$, interval, what does the set $f^{-1}(J)$ --  which
must be in $\mathcal M$ -- look like? 

\vskip .3in

(b) Let $\mu$ and $\nu$ be two measures on $\mathcal M$ defined by 
$$ \mu(A) \,=\, \int_A e^{-x^2}\, dx\,, \quad A \in \mathcal M \,, $$
$$ \nu(A) \,=\, \int_A e^{-x^2+x}\, dx\,, \quad A \in \mathcal M \,.  $$ Show
that $\mu$ is absolutely continuous with respect to $\nu$ and compute
the Radon-Nikodym derivative $d\mu/d\nu$. 
\vskip .1in
Hints: \qquad i) \, Make sure this derivative is $\mathcal M$-measurable. 

\qquad \qquad ii) \, Recall \,  $\cosh(x)\, = \,  \dfrac{ e^{x} + e^{-x}}{2} $  


\newpage 

\item
\vspace{.2in}
% Problem  5
Let $(X, {\cal M}, \mu)$ be a measure space. Let $0 < p <1 $ and let $q$ be such that $\frac{1}{p} + \frac{1}{q} =1$.  Show that if $f$ and $g$  are positive functions then 
\[
\int f g d\mu  \,\ge \,  \left( \int  f^p d\mu\right)^{1/p} \left( \int  g^q d\mu\right)^{1/q}  \,.
\]

{\it Hint:} Use H\"older inequality for some suitable chosen functions ( call them $u$ and $v$.) 









\end{enumerate} 
\end{document}


   \item
\vspace{.2in}
% Problem 2
Let $\mathcal H$ be a Hilbert space and let $S$ be a non-trivial closed subspace of  $\mathcal H$. Let $P_S: \mathcal H \to \mathcal H$ be the orthogonal projection from $\mathcal H$ to $S$. We know then that $P_S$ is linear, and for all $x \in \mathcal H$, $$P_S(x) = y \in S \qquad  \text{and}  \qquad x - P_S (x) = z \in S^{\perp}.$$
Show that: 
\begin{enumerate}

\item\quad  $ \|P_S\|  =1$   \, \, ( $\| \cdot \|$ is the operator norm).

\item \quad $P_S^2 = P_S $   \, \, (left hand side is composition of operators).

\item \quad $P_S^{\ast} = P_S$  \,\,  (ie. $P_S$ is self-adjoint).

\end{enumerate}

\newpage

                                