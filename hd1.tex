
\magnification=1200
\input amstex

\documentstyle{amsppt}
\NoBlackBoxes
%\NoPageNumbers
\def \IR {\Bbb R}
\def \B {\Cal B}
{\catcode`@=11
\gdef\nologo{\let\logo@\empty}
\catcode`@=12}
\nologo

\document

%\head Handout \endhead 


\bigskip 
\head On a Counterexample to Fubini  \endhead

\bigskip 

%The last counterexample to Fubini we covered in class 
%is related to problem $47$. 

Let W be a well ordered set;  
$$j : [0,1] \to W$$ and let 
$Q$ be defined as the set of pairs $(x,y) \in [0,1] \times [0,1]$ 
 such that $j(x)$ preceds $j(y)$ in $W$ (here preceds means with respect to the total order in W). 

We need that 

\roster

\item $Q_x$ should contain all of $[0,1]$ except for a countable set in $[0,1]$.  

\item $ Q^y$  should contain at most a countable set in $[0,1]$. 

\endroster 
Hence note in particular both sections would be Borel sets. 

Then the counterexample follows by considering $f(x,y) = \chi_Q(x,y) $ 
and observing that the integrals over $[0,1]$ w.r.t. Lebegue measure of 
$f_x$ and $f^y$ (which would be Borel functions themselves) were different 
( first one $= 1$; second one $= 0$)

The reason why Fubini doesn't work is because $f$ itself is 
not measurable w.r.t the product Borel sigma algebra

Proving 2) above maybe tricky if $j(x)$ is "too general". One needs to actually assume a few additional things: 

\vskip .1in 

a) Well Ordering Principle: every nonempty set $X$ can be "well ordered". 
(this relies on an equivalent form of Zorn's lemma which states that every partially ordered set has a maximally linearly ordered set i.e. given X there exists a subset E of X for which the partial order in X becomes a total or linear order in E and E is maximal in the sense that no other subset of X that is also 
totally ordered with respect to the same partial order strictly contains E )

\vskip .1in 

b) The Continuum Hypothesis : there is no set whose cardinality is strictly bigeer than that of the natural numbers and strictly smaller than that of the reals.
(In other words if a set is uncountable then its cardinality is bigger or equal than that of the reals).

As a consequence of a) we have the existence of uncountable well order sets. We need then to assume our $W$ above is such a set.

As a consequence of b) and the fact $W$ has cardinality at least that of $[0,1]$ 
we can choose $j(x)$ to be one-to-one function. Now if j(x) is chosen as above in a one-to-one fashion then this guarantees that 1) and 2) above holds as desired. 



\enddocument 






