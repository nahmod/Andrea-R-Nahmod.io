\magnification=1200
\input amstex
\documentstyle{amsppt}
\pageheight{23truecm}
\pagewidth{16truecm}
\parskip 3pt plus 1pt minus 1pt
\hfuzz 0.5cm
\vbadness2000
\baselineskip 13pt
\NoBlackBoxes
{\catcode`@=11
\gdef\nologo{\let\logo@\empty}
\catcode`@=12}
\nologo


%%author macros

\def \f {\frac}
\def \cS {{\Cal S}}
\def \bu {\bullet}
\def\card{\operatorname{card}}
\def\intinfin{\int_{-\infty}^{\infty}}
\def\const{\operatorname{const.}}
\def\dist{\operatorname{dist}}
\def\supp{\operatorname{supp}}
\def\sign{{\rm sgn}}
\def\conpsi{{\text const}_{\psi}}
\def\A{{\Cal A}}
\def\B{{\Cal B}}
\def\C{{\Cal C}}
\def\D{{\Cal D}}
\def\E{{\Cal E}}
\def\F{{\Cal F}}
\def\G{{\Cal G}}
\def\H{{\Cal H}}
\def\I{{\Cal I}}
\def\J{{\Cal J}}
\def\K{{\Cal K}}
\def\L{{\Cal L}}
\def\M{{\Cal M}}
\def\N{{\Cal N}}
\def\Q{{\Cal Q}}
\def\P{{\Cal P}}
\def\R{{\Cal R}}
\def\S{{\Cal S}}
\def\T{{\Cal T}}
\def\V{{\Cal V}}
\def\W{{\Cal W}}
\def\IC{{\Bbb C}}
\def\IH{{\Bbb H}}
\def\II{{\Bbb I}}
\def\IJ{{\Bbb J}}
\def\IN{{\Bbb N}}
\def\IP{{\Bbb P}}
\def\IQ{{\Bbb Q}}
\def\IR{{\Bbb R}} 
\def\IS{{\Bbb S}}
\def\real{{\Bbb R}}
\def\IT{{\Bbb T}}
\def\IW{{\Bbb W}}
\def\IZ{{\Bbb Z}}
\def\zed{{\Bbb Z}}
\def\tee{{\Bbb T}}

\def\gA{{\frak A}}
\def\gB{{\frak B}}
\def\gF{{\frak F}}
\def\gG{{\frak G}}
\def\gH{{\frak H}}
\def\gN{{\frak N}}
\def\gP{{\frak P}}
\def\gT{{\frak T}}
\def \gg{{\frak g}}
\def\nat{{\Bbb N}}
\def\que{{\Bbb Q}}
\def\real{{\Bbb R}}
\def \tq {{\tilde q}}
\def \tp {{\tilde p}}
\def\so{\text{\rm SO}}
\def \inv {|\nabla|^{-1}}
\def\bR{\text{\bf R}}

\overfullrule=0pt

\def\varep{\varepsilon}
\def\ch{\raise 0.3ex\hbox{$\chi$}\kern-.15em}


\def\lto{\longrightarrow}

\def\conj{\overline}

\def\myskip{\noalign{\vskip5pt}}

\def\bmatrix#1{\left[ \matrix #1\endmatrix \right]}



%%endauthormacros



\subhead Basic Littlewood-Paley theory \endsubhead


Let $f(x)$ be a function on $\IR^n$ and $\hat{f}(\xi)$ its Fourier transform.


Consider $m(\xi)$ to be a non-negative radial bump function supported on the ball 
$|\xi |\leq 2$ and equal to $1$ on the ball $|\xi| \leq 1$. 

\vskip .8in

Then for each integer $k$ let  $P_k(f)$ be the {\it  Littlewood-Paley} 
projection operator  onto {\it frequencies}  
$|\xi| \lesssim 2^k$. This is defined by 
$$ \widehat{ P_k (f) }(\xi) := m(2^{-k} \xi ) \hat{f}( \xi ).  $$ 

$\bullet$ $ P_k \to 0$ as $ k \to -\infty$ and $P_k \to I$ as $k \to \infty$ in any reasonable sense- e.g. $L^2$

\vskip .2in 

$\bullet$ The function $P_k(f)(x)$ is a (smoothed) average of $f$ localized to {\it physical scales } $\lesssim 2^{-k}$. By the uncertainty principle one expects $P_k(f)$ to be essentially constant at scales much smaller than $2^{-k}$.   

\vskip .2in 

\noindent The operator $Q_k$ is the projection  
onto the {\it frequency annulus} $|\xi| \sim 2^k \,$ given by the formula,
$$ Q_k := P_k  -  P_{k- 1}. $$

\noindent Hence $ \psi({\xi}) := m( \xi ) - m (2 \xi ) $ is supported 
on the annulus $ 1/2 \le |\xi | \le 2 $,  for all  
$\xi \neq 0,$ \, $ \sum_{k \in \IZ} \psi(2^{-k} \xi) \equiv 1$, and 
$$ \widehat{ Q_k (f)}( \xi) = \psi( 2^{-k} \xi ) \hat{f}(\xi). $$

\vskip .8in



$\bullet$ The L-P projections are bounded operators 
in all the Lebesgue spaces. In fact,  
$Q_k$ is given by a convolution kernel whose $L^p$-norm equals $2^{ ( k n )( 1 - 1/p)}$, \, $1 \le p \le \infty$. 

In particular its $L^1$- norm is identically $1$ for all $ k \in \IZ$. 

\vskip .1in 

$\bullet$  Note that  \quad $ Q_k(f)(x) = P_{k+10} Q_k(f) $  

i.e. it is essentially constant on physical scales $<< 2^{-k}$. 

\vskip .1in 

$\bu$ On the other hand  \quad $ P_{k-10} Q_k(f) \equiv 0 $ 

i.e. it has mean zero at scales $\lesssim 2^{- (k-10)}$. 
\vskip .2in 

$\bu$ On a ball in physical space of radius $O( 2^{-k})$ the function 
$Q_k(f)$ is {\it smooth} at physical scales $<< 2^{-k}$ and contains 
about $O(1)$ oscillations.  In fact, all moments of $Q_k(f)$ vanish: \quad   
$\partial^{\alpha}_{\xi} ( \widehat{ Q_k(f)})(0) \ = \ 0 $.
\vskip .2in


By telescoping the series we have the {\it Littlewood-Paley decomposition}
$$ f \ =: \  \sum_{k \in \IZ}\  Q_k (f) \quad \text{ in the sense of }\,   L^2$$ or for  any locally integrable function with decay at infinity.

$\bu$ We have thus written $f$ as a superposition of functions $Q_k(f)$, each of which has frequency of magnitude $\sim 2^k$. 

{\it Lower values} of $k$ represent {\it low frequency} components of $f$. 
{\it Higher values} represent {\it high frequency} components.

\vskip .1in 

$\bullet$ How are the L-P pieces $Q_k(f)$ of a function related to the function $f$ itself ? 

When $p=2$, by construction and Plancherel, 
$$ || f ||_2  \sim \bigl( \sum_k \| Q_k (f) \|_2^2 \bigr)^{1/2} \sim \Vert \bigl( \sum_k | Q_k(f)(\cdot)|^2 \bigr)^{1/2} \Vert_2 $$

The {\it function} $S(f) (x) \ =: \ \bigl( \sum_k \, | Q_k(f)(x)|^2 \, \bigr)^{1/2} $ is known as the Littlewood-Paley {\it Square Function}.  In general,

\proclaim{Littlewood-Paley Inequality}. For any $1 < p < \infty$ we have 
$$ \| S f \|_p \sim || f ||_p $$ 
\endproclaim 
\noindent Proof is standard harmonic analysis and follows in a starightforward fashion from Calder\'on-Zygmund theory. 

This gives a nice characterization of the Lebesgue spaces in terms of very friendly building blocks.  Much more is actually true. 

\vskip .2in 

\subhead Advantage of this for PDE's \endsubhead

$$ \| \nabla Q_k(f)\|_p  \sim  2^k  \| Q_k(f)\|_p \, , \, \qquad   1 \le p \le \infty $$ 

Roughly, $\nabla$  is multiplication by $ 2\pi i \xi $ 
and $ |\xi| \sim 2^k$  on the support of $ Q_k(f)$. So -morally- 
one can split a derivative as a linear combination of L-P operators.
$$ \nabla f \sim \sum_{k \in \IZ} \, 2^k \,  Q_k(f) $$

$\bu$ The effect of a derivative on a function $f$ is to {\it accentuate} the {\it high frequencies} and {\it diminish} the {\it low frequencies} 

A similar principle applies to other differentiation or pseudodifferential 
opertors ($\Delta$,  $\Delta^{-1}$ or `powers' of them). 

$\bu$ Thus Littlewood-Paley is nicely adapted to dealing with spaces which 
combine $L^p$-type norms with derivatives.  
{\it Sobolev spaces}, {\it Besov spaces}, {\it H\"older spaces, etc}. 

For example, one can very simply characterize the Sobolev norm using L-P decompositions:
$$ \| f \|_{W^{s, p}} \sim \| f \|_p + \| \bigl( \sum_k | 2^{k s}  Q_k (f) |^2 \bigr)^{1/2} \|_p $$

\subhead{Nice Remark} \endsubhead  Although
$W^{n/2, 2}(\IR^n)$ fails to embed into $L^\infty(\IR^n)$  it is 
true on L-P pieces :
$$ \| Q_k(f) \|_{L^\infty}  \lesssim 2^{ \frac{n}{2}\, k} \| f\|_2 $$ 
On L-P pieces:  BMO norm $\sim$ $L^\infty$ norm. 
\vskip .2in 

\subhead Products and Product Estimates \endsubhead 


Let $f$ and $g$ be two nice functions. By splitting them using Littlewood-Paley decompositions we analyze  {\it bilinear } expressions such as  the pointwise product $B(f, g)(x) = f(x) g(x) $. 

$\bu$ Observe that $\widehat{(f\,g)}(\xi) = \int \widehat {f}(\xi - \eta) \widehat{g} (\eta) \, d\eta $ whence $$\text{ supp } \widehat{(f\, g)} \subseteq \text{ supp } \widehat{f} + \text{ supp } \widehat{ g } $$

$$\align f g  &=  \sum_{k, j} Q_k(f) Q_j(g) = \sum_{k \ge j} Q_k(f) Q_j(g) +  \sum_{ k < j} Q_k(f) Q_j(g) \\
&= \sum_{k \in \IZ, m \ge 0} Q_k(f) Q_{k-m}(g) + \sum_{j \in \IZ, m > 0} Q_{j-m}(f) Q_j(g) \endalign $$ Further, we can split 
$$\align  f g  &\  = \  \sum_l Q_l ( f g ) \ =  \\ = \sum_{l, k} \sum_{m \ge 0} &  Q_l ( Q_k(f) Q_{k-m}(g)) + \\  &+ \ \sum_{l,j} \sum_{m > 0} Q_l(  Q_{j-m}(f) Q_j(g)). \endalign $$  But by inspecting the Fourier supports we find that :
$$ \text{ (1) \ \ \ \ supp} \, (\widehat{ Q_k(f) Q_{k-m} (g)}) \subseteq \{ | \xi| \lesssim 2^{k}\}. $$ Hence  $$Q_l( Q_k(f) Q_{k-m} (g)) = 0 \text{ unless }  k \ge l$$ 
\newpage 
  
$$\text{(2) \ \ supp}\, (\widehat {Q_k(f) Q_{k-m} (g)})   \cap  \{ |\xi| << 2^{k-m} \} = \emptyset \text{ if } m > 5 .$$  Hence, 
$$\align  Q_l( Q_k(f) Q_{k-m} (g)) \ & = \ 0 \quad \text{ unless }: \\
& l =k  \text{ and } m > 5  \qquad \text{ \underbar{\bf or}}   \\
& l < k  \text{ and } 0\le m \le 5 \endalign $$ All in all we {\bf only} have {\bf three types } of  sums :

$$\align f g &=  \sum_l \sum_{m \ge 5} Q_l ( Q_l(f) Q_{l-m}(g)) + \\  
&+ \ \sum_{m=0}^5 \sum_l  \sum_{ k > l} Q_l(  Q_{k}(f) Q_{k-m}(g)) \\ 
&+\,  \sum_l \sum_{m \ge 5}   Q_l ( Q_{l-m}(f) Q_{l}(g))  \\  
&+ \ \sum_{m=0}^5 \sum_l  \sum_{ j > l} Q_l(  Q_{j-m}(f) Q_{j}(g)) \endalign $$

\vskip .2in 

This could be re-written as well as : 

$$\align & f g  = \sum_{\ell}  P_{\ell+1}(f) P_{\ell+1}(g) -  P_{\ell}(f) P_{\ell}(g) \\ 
& =  \sum_{\ell}  Q_{\ell}(f) P_{\ell}(g) + \sum_{\ell}  P_{\ell}(f) Q_{\ell}(g)\\
& \quad +  \sum_{\ell}  Q_{\ell}(f) Q_{\ell}(g) \endalign  $$

\vskip .3in 

Paraproduct \,  + \, Paraproduct  \, +  \, Diagonal .

\vskip .2in

$\bu$ In various applications the {\it high-low} interactions are 
`easily dealt with'   ( $\sim$ paraproducts). It is the {\it high-high} interactions what usually may account for energy cascade effects; they are subtler to analyze. 


 


\enddocument


























