\magnification=1200
\input amstex

\documentstyle{amsppt}
\NoBlackBoxes
%\NoPageNumbers
\pagewidth{16.5truecm}
\pageheight{22truecm}

{\catcode`@=11
\gdef\nologo{\let\logo@\empty}
\catcode`@=12}
\nologo
%\hcorrection{.3in}

\centerline{ \bf  M534H   HOMEWORK-- Spring 2017}
\vskip .1in
\centerline{ Prof. Andrea R. Nahmod }
\vskip .2in

\document

$\bullet$\, {\bf{Set 1. Due date:  Thursday February 9th }}
\vskip .2in

1. Let $(x, y) \in \Bbb R^2$. Determine which of the following differential operators is {\it linear} :
\medskip 
\roster
\item  $\Cal L (u) = 2 u_x - x u_y$
\smallskip 

\item  $\Cal L (u) =  4 u_x + u u_y$
\smallskip 

\item  $\Cal L (u) = u_x +  u_y^2 $
\smallskip 

\item  $\Cal L (u) = 1 - u_x + 2  u_y$
\smallskip 

\item  $\Cal L (u) = \cos(y) u_x +  \sqrt{x^2 +4}  \, u_{xy} + \arctan(2 x) u $

\endroster 

\bigskip 

2.  For each of the following equations, state the {\it order} and whether is {\it nonlinear}, {\it linear and homogeneous} or {\it linear and inhomogeneous}. Explain/justify your answer.
\medskip

\roster
\item  $  u_t -  2 u_{xx} +1 =0 $
\smallskip 

\item  $  u_t -  2 u_{xx} + x u =0$
\smallskip 

\item  $  u_{tt} -  2 u_{xx} + x^4 =0$
\smallskip 

\item  $ u_t -   u_{xxt} + u u_x =0$
\smallskip 

\item  $ i u_t -   2 u_{xx} + \frac{u}{3x} =0$
\smallskip 

\item  $ 3 u_y + 2 e^{-y} u_{x} =0$

\smallskip 

\item  $ u_t - u_{xxxx}  + \sqrt{ 1 + u^2} =0$
\endroster 

\vskip  .2in

3. Let $g$ be a given function and let $u_1$ and $u_2$ be two solutions to the linear inhomogeneous differential equation $\Cal{L}(u) = g$. 
Prove that $u_1 - u_2$ solves the linear homogeneous differential equation $\Cal{L}(u) = 0$. 

\vskip  .2in

4.  Let $g \in C^2([0, \pi])$ such that $g(0)=g(\pi) =0$. Use the method of separation of variables to find the solution to:  
\vskip .1in 

$u_t - 2 u_{xx} =0\, $ in $0< x< \pi$ with Dirichlet boundary conditions $u(0,t)\, = \,u(\pi, t) \,=\,0$ and Cauchy data $u(x, 0)= g(x)$

\bigskip
{\it Hint}. First find the general (series) solution to $u_t - 2 u_{xx} =0\, $ in $0< x< \pi$ satisfying $u(0,t)\, = \,u(\pi, t) \,=\,0$. 
\vskip  .2in
\newpage
5. Use the method of separation of variables to find the general solution to: 
\vskip .1in 


\quad $u_t - 4 u_{xx} =0 $ in  $0<x< 1$ with Neumann boundary conditions $u_x(0,t) = u_x (1,t) =0$. 

\vskip  .2in


6.  Let $g \in C^2([0, 1])$ such that $g(0)=g(1) =0$.  Consider the Dirichlet  IVB problem 
\vskip  .1in
$u_t -  u_{xx} + 3 u= 0\, $ in $0< x< 1$  with $u(0,t)\, = \,u(1, t) \,=\,0$ and  $u(x, 0)= g(x)$  \quad  $(\ast)$
\vskip  .1in
a)  Consider the change of variables $u(x,t) = e^{-3 t} v (x, t)$ and prove that $v$ solves 
\vskip  .1in
$v_t -  v_{xx}= 0$ \, in $0< x< 1$  with $v(0,t)\, = \,v(1, t) \,=\,0$ and $v(x, 0)= g(x)$.   \quad  $(\dagger)$
\vskip  .1in
b)  Use the method of separation of variables to find  $v$,  the solution to $(\dagger)$.
\vskip .1in
c)  Use a) and b) to find $u$, the solution to  $(\ast)$.

\vskip  .3in

$\bullet$\, {\bf{Set 2. Due date:  Thursday February 23rd}}

\vskip  .2in
1)  Let $\phi : \Bbb R \to \Bbb R$ be a periodic function with period $p$; that is $\phi(x + p) = \phi(x), \, \, \forall x \in \Bbb R$. Assume that 
$\phi$ is integrable on any finite interval. 
\smallskip

(a) Prove  that for any $a, b \in \Bbb R$ 
$$  \int_a^b \phi(x) d x  = \int_{a+ p}^{b+p} \phi(x) dx = \int_{a- p}^{b - p} \phi(x) dx $$
\smallskip

(b) Prove  that for any $a \in \Bbb R$ $$  \int_{-p/2}^{p/2} \phi(x+a) d x  = \int_{-p/2}^{p/2} \phi(x) dx = \int_{- p/2 +a }^{p/2+a} \phi(x) dx $$

\smallskip 
In particular we have that $\int_a^{a+p}  \phi(x) \, dx$ does not depend on $a$, as we discussed in class. 
\vskip  .2in

2)  Let $x \in (-\ell, \ell)$. Prove that the family  $\{ 1, \cos( \dfrac{n\pi x}{\ell}),  \sin(\dfrac{n\pi x}{\ell}) \}_{n \geq 1}$ satisfies: 
\medskip
$$(i) \qquad \int_{-\ell}^{\ell}  \cos( \dfrac{n\pi x}{\ell}) \sin(\dfrac{m\pi x}{\ell}) \, dx \, =\, 0   \, \, \, \text{for any } \,  n, \, m \in \Bbb Z.$$
\smallskip
$$(ii) \qquad \int_{-\ell}^{\ell}  \cos( \dfrac{n\pi x}{\ell}) \cos(\dfrac{m\pi x}{\ell}) \, dx \, =\, 0   \, \, \, \text{ whenever } \, n\neq m  \, \in  \Bbb Z.$$
\smallskip
$$(iii) \qquad \int_{-\ell}^{\ell}  \sin( \dfrac{n\pi x}{\ell}) \sin(\dfrac{m\pi x}{\ell}) \, dx \, =\, 0   \, \, \, \text{ whenever } \, n\neq m  \, \in  \Bbb Z.$$
\smallskip
$$(iv) \qquad \int_{-\ell}^{\ell}  1 \, . \, \cos( \dfrac{n\pi x}{\ell}) \, dx \, =\, 0 \, =\int_{-\ell}^{\ell}  1 \, . \, \sin( \dfrac{n\pi x}{\ell}) \, dx   \, \, \, \text{ for any} \, n \in  \Bbb Z.$$
\smallskip
$$(v) \qquad \int_{-\ell}^{\ell}  \, \cos^2( \dfrac{n\pi x}{\ell}) \, dx \, =\, \ell\, =\int_{-\ell}^{\ell}  \, \sin^2( \dfrac{n\pi x}{\ell}) \, dx   \, \, \, \text{ for any} \, n \in  \Bbb Z.$$
\smallskip
$$(vi) \qquad \int_{-\ell}^{\ell}  1^2 \, dx \, =\, 2\ell. $$
\smallskip
{\it Hint.} Use the trigonometric identities. 
\vskip  .2in

3) Consider diffusion inside an enclosed circular tube. Let its length (circumference) be $2 \ell$. Let $x$ denote the arc length parameter where $-\ell \leq x \leq \ell$. Then the concentration of the diffusing substance satisfies:
\vskip  .1in
$$ \cases u_{t} \, =\,  k u_{xx}  \qquad -\ell < x < \ell , \\
u(-\ell, t) \,=\, u(\ell, t)\quad \text{and} \quad u_x(-\ell, t) \,=\, u_x(\ell, t). \endcases $$ 
\vskip .1in
These are called {\it periodic boundary conditions}. {\underbar{Show that:}}
\vskip .1in
a) the eigenvalues are  \, $ (\dfrac{n \pi}{\ell})^2, \, n=0, 1, 2, 3, \dots $
\vskip .1in
b) the concentration $u(x,t)$ is given by: 

$$u(x, t)\,=\, \frac{1}{2} A_0 \, +\, \sum_{n=1}^{\infty} \bigl( A_n  \, \cos( \dfrac{n\pi x}{\ell})  \, +\, B_n \sin( \dfrac{n\pi x}{\ell}) \bigr)\, e^{- \frac{n^2 \pi^2}{\ell^2} k t}  $$


\vskip  .2in

4) Consider the function $\phi(x) \equiv 1$ defined on $[0, \ell]$. 
\smallskip
a) Find its {\underbar{Fourier sine} series in $(0, \ell)$.
\vskip  .1in
b) Find its {\underbar{Fourier cosine} series in $(0, \ell)$.

\vskip  .2in

5) Consider the function $\phi(x) = x $ defined on $[0, \ell]$. {\underbar{Prove that}}:
\smallskip
a) its Fourier sine series in $(0, \ell)$ is given by
$$ x \, =\, \frac{2 \ell}{\pi} \bigl(\sin{\frac{\pi x}{\ell}}  -   \frac{1}{2} \sin{\frac{2 \pi x}{\ell}}  + \frac{1}{3} \sin{\frac{3\pi x}{\ell}} \, - \, \dots   \bigr).  $$

\vskip .1in
b) its Fourier cosine  series  in $(0, \ell)$ is given by
$$ x \, =\, \frac{\ell}{2} \, -\, \frac{4 \ell}{\pi^2}  \bigl(\cos{\frac{\pi x}{\ell}}  +  \frac{1}{9} \cos{\frac{3 \pi x}{\ell}}  + \frac{1}{25} \cos{\frac{5\pi x}{\ell}} \, + \, \dots   \bigr).  $$
 
 \vskip .2in 
 
6) Consider now $\phi(x) = x $ defined on $[-\ell, \ell]$. Find its full Fourier series  in $(-\ell, \ell)$.
\vskip .2in
Compare your answer here ( valid on $(-\ell, \ell)$ ),  with that found in 5a) (valid on $(0, \ell)$)  while noting that $\phi$ is an {\it odd function} on $(-\ell, \ell)$. 
 \vskip .2in 
 
 7) Using the corresponding definitions of {\it odd} and {\it even} functions, \underbar{prove} that:
 \smallskip
 
 a) if $\phi$ is an {\it odd function} then \, \, $\int_{-\ell}^{\ell}  \phi(x) \, dx  \,=\, 0. $
 \smallskip
 b)  if $\phi$ is an {\it even function} then \, \, $\int_{-\ell}^{\ell}  \phi(x) \, dx  \,=\,  2 \int_{0}^{\ell}  \phi(x) \, dx$
 
  \vskip .2in
 8) Use Problem 7) and the definitions of what the coefficients are in each case to prove that:
 \smallskip
a) if $\phi$ is an {\it odd function} its full Fourier series on $(-\ell, \ell)$ has only {\underbar{sine}} terms.
\smallskip
b)  if $\phi$ is an {\it even function} its full Fourier series on $(-\ell, \ell)$ has only {\underbar{cosine}} terms.
\vskip .1in
{\underbar{\it Hint.}}  Do not use the series directly. Use the formula for the coefficients and Problem 7) to show that certain 
coefficients vanish in each case.

  \vskip .3in
  
  9) This problem tell us how to change of variables to reduce the homogeneous diffusion equation on $(0, \ell)$ with constant but non-zero Dirichlet  boundary conditions to a homogeneous diffusion equation on $(0, 1)$ with zero Dirichlet conditions.  More precisely: let $k >0$ be fixed and suppose that real constants $u_{0},  u_{\ell}$ and a function $g \in C^2([0,\ell])$ are given. Consider the problem:
  $$(\dagger)\quad  \qquad  \cases u_{t} \,- \,  k u_{xx} \, =\, 0,   \qquad  0< x < \ell, \, \, \, t>0 \\
  u(x, 0) \, =\, g(x), \quad   0< x < \ell \\
u(0, t) \,= \, u_{0} \quad \text{and} \quad u(\ell, t) \,=\, u_{\ell} \endcases $$ 

\vskip .1in

a) Show that if $u(x, t)$ is a solution to  $u_{t} \,- \,  k u_{xx} \, =\, 0$  then the function $ u_{\gamma}(x, t) := u(\gamma x, \gamma^2 t)$ also solves $(u_{\gamma})_{t} \,- \,  k  (u_{\gamma})_{xx} \, =\, 0$   
\vskip .1in
b) Let  $$\tau: = \frac{\ell^2}{k}, $$ set $$y:= \frac{x}{\ell}, \qquad  s:= \frac{t}{\tau}$$ and define $$v(y, s):= u(\ell y, \tau s),\qquad 0 < y <1.$$ {\underbar{Prove}} that $v(y, s)$ is a solution to 
  $$(\dagger\dagger)\quad  \qquad  \cases v_{s} \,- \,  v_{yy} \, =\, 0,   \qquad  0< y < 1, \, \, \, s>0  \\
  v(y, 0) \, =\, g(\ell y), \quad 0< y<1 \\
v(0, s) \,= \, v_{0} \quad \text{and} \quad v(1, s) \,=\, v_{1} \endcases $$ where $v_0 = u_0$ and $v_{1} = u_{\ell}$. 
\vskip .1in
c)  Let us denote by $\tilde g(y) := g(\ell y)$ and define $$w(y, s):= \frac{ v(y, s) - v_0}{v_1 - v_0}  \, - \, y, \quad \text{ and } \quad  G(y):=  \frac{\tilde g(y) - v_0}{v_1 - v_0} \, -y.$$ {\underbar{Prove}} that $w(y,s)$ solves:
$$(\ast)\quad  \qquad  \cases w_{s} \,- \,  w_{yy} \, =\, 0,   \qquad  0< y < 1, \, \, \, s>0 \\
  w(y, 0) \, =\, G(y), \quad 0<y<1  \\
w(0, s) \,= \, 0 \quad \text{and} \quad w(1, s) \,=\, 0, \endcases $$  
\vskip .1in

The idea then is that once we find $w(y, s)$, the solution to $(\ast)$, we can undo the definitions and change of variables above to obtain $u(x, t)$, the solution to $(\dagger)$.
  
 \vskip .3in
  
$\bullet$\, {\bf{Set 3. Due date: Thursday March 2 }}
  \bigskip
  
{\underbar{Problem I}}:  a) {\underbar{From S. Salsa 3rd Edition (Chapter 2 pg 109)} :   Do problem  2.2 

\bigskip

b) {\bf Do not turn in this part; I'll explain this in class.}
 Assume that $g \in C^2( [0, L])$ whence  $u$ is continuous  on $[0, L] \times [0, \infty)$.  Interpret the problem supposing $u$ is the concentration of a substance under diffusion and explain why as $t \to \infty$, the solution 
$$u(x, t) \longrightarrow   M:= \frac{1}{L}   \int_0^L \, g(y) \, dy,  \qquad \text{for all} \quad 0 < x < L. $$
\underbar{Hints.} Recall here we have zero {\bf Neumann} boundary conditions.  Prove first that $\frac{d}{dt} \int_0^L u(x, t) \, dx \, =\, 0$, whence $\int_0^L u(x, t) \, dx $ is constant in time and that by the continuity of $u$, \,  $\int_0^L u(x,t)\, dx \longrightarrow  \int_0^L g(x)\, dx$ as $ t \longrightarrow 0^+$. )
  
 \bigskip
 
  c) Assume that $g \in C^2( [0, L])$ with $g^{\prime}(0) = 0 = g^{\prime}(L)$ and consider the even extension $g_{\text{even}}$ of $g$ on $[-L, L]$ --in particular $g_{\text{even}}$ is also $C^2([-L, L])$. Show then that the {\it full Fourier series} of $g_{\text{even}}$ on $[-L, L]$ reduces to just a {\it cosine} Fourier series of the form
  $$  \frac{A_0}{2}  \,  + \, \sum_{m=1}^\infty \, A_m  \cos( \frac{m \pi x}{L} ) $$ where 
  $$  \frac{A_0}{2} \,=\, M, \qquad \text{and} \qquad  A_m=  \frac{2}{L} \int_0^L \, g(x)  \cos( \frac{m \pi x}{L} ) \, dx $$
 
  \bigskip

 
{\underbar{Problem II}}:  a) Prove that the solution $u$ to Problem 4) in Set 1 is continuous on $[0, L] \times [0, \infty)$. 
     \medskip
     b) Use the maximum principle to prove that said solution is unique. 

  \bigskip
  
  
{\underbar{Problem III}}:  Use the energy method to show that the solution to problem 2.2 (from part a) Problem I above) is unique. 
  

\vskip .3in
  
$\bullet$\, {\bf{Set 4. Due date: Thursday March 23rd }}
  \vskip .2in
%The following problems are about uniqueness, maximum principle and energy method.   
  %\vskip .2in
1) Supposed that $u(x,t):= 1 - x^2 - 2 k t$ is a solution to the diffusion $u_t - k u_{xx} =0$. Find the locations of its maximum and of its minimum in the closed rectangle ${\overline Q_T} = [0, 1] \times [0, T]$
  \vskip .2in
2)  Let $u(x,t)$ be a solution to 

  $$\cases u_{t} \,- \,  u_{xx} \, =\, 0,   \qquad  0< x < 1, \, \, \, t>0  \\
  u(x, 0) \, =\, \sin(\pi \, x) , \quad 0< x<1 \\
u(0, t) \,= \, 2 t \,e^{1-t} \quad \text{and} \quad u(1, t) \,=\, 1 - \cos(\pi\, t) \endcases $$
  that is continuous on the closed half strip $\overline S = [0, 1] \times [0, \infty)$.
  
 \bigskip
 a) Prove that $u(x, t) \geq 0$. Show your analysis along  $\partial_p S$, the parabolic boundary of $S$.
 
 \medskip
 b) By carefully examining the values along the parabolic boundary of $S_{\frac{1}{8}} = (0, 1) \times (0, \frac{1}{8})$ determine an upper bound for 
 $u(\frac{1}{2}, \frac{1}{8})$ (that is,  an explicit constant $A>0$ such that $u(\frac{1}{2}, \frac{1}{8}) \leq A$). 
 
Proceeding similarly find an upper bound for  $u(\frac{1}{2}, 3)$ ( consider now  $\partial_p S_{3}$).
 
  \vskip .2in
  
  3)  Let $u(x, t)$ be a continuous solution to the diffusion equation $u_t - u_{xx}=0$ in $ [0, \ell] \times  [0, \infty)$. 
  \medskip 
  a) Let $M(T)$ be the maximum of $u(x,t)$ in the closed rectangle $[0, \ell] \times [0, T]$. Does $M(T)$ increase or decrease as a function of $T$? 
  \medskip
  b) Let $m(T)$ be the minimum of $u(x,t)$ in the closed rectangle $[0, \ell] \times [0, T]$. Does $m(T)$ increase or decrease as a function of $T$? 
  
    \vskip .2in
    
    4)  Let $u(x,t)$ be a smooth solution to 
    
 $$\cases u_{t} \,- \,  u_{xx} \, =\, 0,   \qquad  0< x < 1, \, \, \, t>0  \\
  u(x, 0) \, =\, 4 x (1-x)  , \quad 0< x<1 \\
u(0, t) \,= \, 0 \quad \text{and} \quad u(1, t) \,=\, 0. \endcases $$ 
a) Use the max/min principle to show  $0 \leq u (x, t) \leq 1$ for  $0 < x<1 $ and $t >0$.

\medskip

b) Show that  $u(x,t) =  u(1-x, t)$ for all $ t \geq 0$ and $0 \leq x \leq 1$.
\smallskip
\noindent \underbar{Hint.} Do not solve explicitly. Rather prove that $u(1-x, t)$ also solves the equation and then apply the uniqueness theorem.
  \medskip
  c) Use the {\it energy method} to show that $$ E(u)(t):= \int_0^1  [u(x,t)]^2 \, dx   $$ is a strictly decreasing function of $t$. 
     
      \vskip .2in
      5)   Prove the Comparison Principle: that is prove part a) of Corollary 2.5 p. 38 (Salsa, 3rd edition). (use Theorem 2.4 (max. pple) ).
      
      \vskip .2in
      6) The purpose of this exercise is to show that the maximum principle is {\bf not} true for the variable coefficients heat equation
      $u_t - x\, u_{xx} =0$.
      \medskip
      a) Verify that $u(x,t) = - 2xt - x^2$ is a solution. 
            \medskip
      b) Find the location of its maximum in the closed rectangle $\{ -2 \leq x \leq 2, \, 0\leq t \leq 1 \}$.
            \medskip
            c) Where precisely does our proof of the maximum principle break down for this equation?     
      
      \vskip .2in
      
7)      This problem pertains the heat equation on the whole real line ($\&$ 2.3). Let us introduce Gauss' {\it error function} 
$$ \text{erf}(x) \, :=  \, \frac{2}{\sqrt{\pi}}\, \int_0^x  e^{-z^2} dz .$$ 

a) Prove that all solutions to the equation 
$u_t - k u_{xx} =0$ of the form  $ u(x, t) = v(\frac{x}{\sqrt{t}})$, for $x \in \Bbb R$ and $t >0$ have the form $$(\dagger) \qquad \quad  u(x, t) = C_1 + C_2 \, \text{erf}(\frac{x}{\sqrt{4k t}}). $$

\medskip

b) Prove that by choosing $C_2$ suitably in $(\dagger)$ the fundamental solution $\Gamma_k(x,t) = u_x(x, t)$.
\medskip 
\underbar{Hint.} For a):   First change variables $y:= \frac{x}{\sqrt{t}}$, whence $$\frac{\partial y}{\partial t}=  -\frac{x}{2 t \sqrt{t}}, \quad
\frac{\partial y}{\partial x}=  \frac{1}{\sqrt{t}}, \quad \text{and} \quad \frac{\partial^2 y}{\partial x^2}= 0. $$ Then prove that if $u(x,t)$ is a solution to the heat equation as above then $v(y)$ solves the ODE 
$$ \frac{y}{2k} v^{\prime}(y) \, +\, v^{\prime\prime}(y) \, =\, 0$$ which can be viewed as a {\it first order} ODE for   w= $v^{\prime}(y)$. Solve then first $ \frac{y}{2k} w(y) \, +\, w^{\prime}(y) \, =\, 0$ and then by further integration find $v(y)$ expressed in terms of Gauss' {\it error function}. Recall
finally that $u(x, t) = v(\frac{x}{\sqrt{t}}).$

      \vskip .2in
      
      8) ( Differentiation Theorem) Suppose that $f(x,t)$ and $\frac{\partial f}{\partial t} (x, t)$ are continuous functions of $x$ and $t$ on
      the rectangle $[A, B] \times [ S, T] $. Suppose that $a(t), b(t)$ are differentiable functions of $t$ on $[S,T]$ and that the space interval $[A, B]$ contains
      the union of all intervals $[a(t), b(t)]$ as $t$ varies on $[S, T]$.  
      \, (A simple \underbar{example} of this would be the interval  $[A, B] = [0, 1]$ in $x$, the interval $[S, T] = [0, 1]$ in $t$ and the functions $a(t)=0$ and $b(t)= t$.) 
      
      Consider $$ I(t):= \int_{a(t)}^{b(t)} \, f(x, t) \, dx. $$ Prove that then 
      $$ \frac{dI}{dt} \, =\, \int_{a(t)}^{b(t)} \, \frac{\partial f}{\partial t} (x, t) \, dx \, +\, f( b(t), t)\, b^{\prime}(t)  \, -\, f( a(t), t) \, a^{\prime}(t). $$
 
\vskip .1in
      
      \underbar{Hint.} Think of $I(t)$ as $g(t, a(t), b(t))$ where $g$ is a function of three independent variables and apply chain rule. Differentiation under the integral sign is possible since both $f(x,t)$ and $\frac{\partial f}{\partial t} (x, t)$ are continuous. 
\vskip .1in

We will use this differentiation theorem in class. 

\newpage

$\bullet$\, {\bf{Set 5. Due date: To be turned in on Midterm date}}

\vskip .2in

{\underbar{Problem 1}} a) Show that the fundamental solution on $\Bbb R$ given by,  $$\Gamma_k(x,t):= \frac{1}{\sqrt{4 \pi kt}}  e^{\frac{-x^2}{4kt}}$$ solves the heat equation $ u_t - k u_{xx} = 0$.


\vskip .1in

b) Show that $$ \int_{-\infty}^{\infty} \, \Gamma_k(x,t) \, dx \, =\, 1 \qquad \text{ for all} \quad t >0.$$
\medskip
\underbar{Hint} Change variables $y:= \frac{x}{\sqrt{4kt}}$ and use that $\int_{-\infty}^\infty  e^{-y^2}  \, dy = \sqrt{\pi}$
\medskip
c) Use b) and another simple change of variables to show that $$ \int_{-\infty}^{\infty} \, \Gamma_k(x-z ,t) \, dz \, =\, 1 \qquad \text{ for all} \quad t >0.$$

\vskip .2in
{\underbar{Problem 2} a) Show that the function $ u(x, t) = e^{-k t} \sin(x) $ solves $ u_t - k u_{xx} = 0$.
\medskip
b) Find a relationship between the constants $a$ and $b$ so that $u(x,t)= e^{-a t} \cos(b x) $ is a solution to $ u_t - k u_{xx} = 0$ (assume $\cos(bx) \neq 0$). 

\vskip .2in
{\underbar{Problem 3}:  Consider the Cauchy IVP for the heat equation:
$$(\dagger\dagger) \qquad \cases u_{t} \,- \,  k u_{xx} \, =\, 0,   \qquad  x \in \Bbb R, \, \, \, t>0  \\
  u(x, 0) \, =\, H(x) , \quad x \in \Bbb R \endcases $$
where $H(x)$ is the Heaviside function, $H(x) = 0$ if $x <0$ and $H(x) =1$ for $x \geq 0$.  

Using the general formula given in 7a) find the constants $C_1$ and $C_2$ so that $u(x,t)$  \underbar{also} satisfies the initial condition and hence is a solution to $(\dagger\dagger)$. 

In other words find $C_1$ and $C_2$ so that  for $x<0$ 
$$ 0 = u(x, 0^{+}) = \lim_{t \to 0^{+}}  C_1 + C_2 \, \text{erf}(\frac{x}{\sqrt{4k t}}) $$ 
and for $x  >0$, 
$$ 1 = u(x, 0^{+}) = \lim_{t \to 0^{+}}  C_1 + C_2 \, \text{erf}(\frac{x}{\sqrt{4k t}}) $$
Note in the first case get an integral between $0$ and $-\infty$ while on the second you get an integral between $0$ and $\infty$.

After finding $C_1$ and $C_2$ write down the final expression for $u(x,t)$ solving $(\dagger\dagger)$ in terms of the $ \text{erf}$ function and the explicit constants $C_1$ and $C_2$.


\vskip .2in
{\underbar{Problem 4}: Find the solution to the following Cauchy IVP for the heat equation:
$$\cases u_{t} \,- \,  k u_{xx} \, =\, 0,   \qquad  x \in \Bbb R, \, \, \, t>0  \\
  u(x, 0) \, =\, 2 , \quad x \in \Bbb R \endcases, $$ 
  by using Problem 1c) and that $u(x,t) = (\Gamma_k (\cdot, t) \ast g)(x)$ where $g$ is the initial datum. Check your answer indeed solves the Cauchy IVP above.
   
 
\vskip .2in


{\underbar{Problem 5}:  Let $\Omega \subset \Bbb R^d$ be a connected domain with a smooth boundary $\partial \Omega$. Use the energy method to prove the uniqueness of solutions of the Dirichlet BVP for the Laplace equation: 
$$(\dagger) \qquad \cases \Delta u  \, =\, f , \quad \text{ for}  \qquad  x \in \Omega,  \\
  u\, =\, g, \quad \text{for} \quad x \in \partial \Omega \endcases  $$ 
  
(\underbar{Hint} Consider $w = u - v$ where both $u$ and $v$ solve the same problem $(\dagger)$, multiple $\Delta w$ by $w$ and, as we did in class, integrate by parts -or use the divergence theorem-).

\vskip .2in
{\underbar{Problem 6}: Let  $ \Omega \subseteq \Bbb R^d$ be a smooth domain. A function $v \in C^2(\Omega)$, $ \Omega \subseteq \Bbb R^d$  is said to be {\it subharmonic} if  $\Delta v \geq 0$. 
\smallskip
a) Show that $v(x,y):=  (x^2 + y^2)^2$, \, $(x,y) \in\Bbb R^2$ is subharmonic. Write down your computation of $\Delta v$. 
\medskip
b) Prove that if $u$ is harmonic then $u^2$ is subharmonic. 

\vskip .2in
{\underbar{Problem 7}: Consider the function $u(x,y):=  x^2 - y^2$. 

a) Prove that $u$ is harmonic.
\smallskip
b) Find $$ \frac{1}{ |B_R(0)|} \int_{B_R(0)} \, u(x,y) \, dx dy $$
\underbar{Hint} Use the mean value property (as we saw in class or Theorem 3.4 page 122 of Salsa). Note that 
$\frac{d}{\omega_d R^d}$  or what in class I denoted simply by $\frac{c_d}{R^d}$ amounts to $1$ over the area/volume of
the ball of radius $R$ in $\Bbb R^d$.


\vskip .2in

$\bullet$\, {\bf{Set 6. Due date: Thursday April 20th}}

\vskip .2in
{\underbar{Problem 1}: If $v$ is subharmonic ($\Delta v \ge 0$) then the mean of $v$ over $B_R(x_0)$ is bigger than or equal than $v(x_0)$   (instead of equal). That is 
$$ v(x_0)  \leq \frac{1}{|B_R(x_0)|} \int_{B_R(x_0)} \, v(y) \, d y.$$  Use this fact (together with Pb 6b) in Set 5) to prove that if  $u$ is harmonic on $\Omega = \Bbb R^d$ and 
$$ \int_{\Bbb R^d} u^2(y) \, dy \leq M, $$ 
where $0< M< \infty$ is a constant; then $ u \equiv 0$. 

\vskip .2in
{\underbar{Problem 2}:  Determine for which $\alpha \in \Bbb R$ the function $u(x) = |x|^{\alpha}$ is subharmonic in $\Bbb R^d \setminus \{0\}$.  (\underbar{Hint}. Note that if $u$ is radial,then $\Delta u = u_{rr} + \frac{d-1}{r} u_r $ where $r=|x|$. 

\vskip .2in
{\underbar{Problem 3}: Let $u=u(x,y) \geq 0$ be harmonic on $B_4(0,0) \in \Bbb R^2$ with $u(1, 0)=1$. Using Harnack's inequality find the upper and lower bounds for $u(-1, 0)$.

\vskip .2in
{\underbar{Problem 4}: Let $\Omega$ be a bounded domain in $\Bbb R^d$ and suppose $u \in C^2(\Omega) \cap C(\overline{\Omega})$ is a solution of the equation $\Delta u = u^3 - u$ with $u =0$ on $\partial \Omega$. Show that $|u(x)| \leq 1$ for all $x \in \Omega$

\vskip .2in
{\underbar{Problem 5}:  Do Problem 3.6 Chapter 3 of S. Salsa (p. 175, 3rd ed.).

%Do Problem 3.14 Chapter 3 of S. Salsa (p. 176-177, 3rd ed.).

\vskip .2in
$\bullet$\, {\bf{Set 7. Due date: Thursday April 27th}}
\vskip .2in

{\underbar{Problem 1}}:  Recall in class we showed that the Green's function for the upper half plane $x, y \in \Bbb R^d_{+}=\{ (x_1, \dots,  x_d) : \, x_d >0 \}$ 
is given by  $$ G(x,y): = \Phi(x, y) -  \Phi(x^{\ast}, y), \quad x^{\ast}= (x_1, \dots , -x_d), \quad x, y \in \Bbb R^d_{+} $$
where $\Phi$ is the fundamental solution for the Laplace opertor $\Delta$ in dimension $d$. 
\smallskip

Consider $d=2$ and compute the Poisson kernel, $\frac{\partial G}{\partial {\bold n}}( x, \overline{y})$, where $\overline y = ( y_1, 0)  \in \partial \Bbb R^d_{+}$.  \, Recall $\frac{\partial G}{\partial {\bold n}} = \nabla_y G \cdot {\bold n}$ and $\bold n = ( 0, -1)$ is the outward normal vector to 
$\partial \Bbb R^d_{+}$.

\vskip .2in
{\underbar{Problem 2}}: Follow the idea and computation we did in class to find the Green's function for the unit ball in $\Bbb R^3$ 
to find instead the Green's function for the ball in $\Bbb R^3$ of radius $R>0$. 
\smallskip 

\noindent \underbar{Hints}. Note that now  $x^{\ast}$ will depend on $R$. Also, treat the case $x=0$ separately.
\vskip .2in

{\underbar{Problem 3}}:  Consider the following Poisson equation on the upper half plane $\Bbb R^2_{+}$ with Dirichlet boundary data 
$g \in C(\partial \Bbb R^2_{+})$, 
$$(\dagger) \qquad \cases  \Delta u\, =\, 0,   \qquad  x \in \Bbb R^2_{+},  \\
  u(\overline x) \, =\, g(\overline x) , \quad {\overline x} \in  \partial \Bbb R^2_{+} \endcases, $$ 

Use the representation formula for solutions to the Poisson equation with Dirichlet boundary data in 
conjunction with the explicit formula for the Poisson kernel found in Problem 1 to write down an integral formula for $(\dagger)$.

\vskip .2in

\underbar{The following problems are about the transport equation and the wave equation}
\vskip .2in

{\underbar{Problem 4}}: a) Find the general form for the solution $u=u(x,y), \, x, y \in \Bbb R$ to the transport equation $ 4 u_x - 3 u_y =0$. 
\smallskip
b) Use part a) to find the solution $u$ that satisfies the auxiliary condition $u(0, y) =  y^3$ 

\vskip .2in

{\underbar{Problem 5}}: Consider the wave equation on $\Bbb R$, 
$$(\dagger) \qquad \cases u_{tt} \,- \,  c^2 u_{xx} \, =\, 0,   \qquad  x \in \Bbb R,   t>0  \\
  u(x, 0) \, =\, \phi(x), \quad   u_t(x, 0) = \psi (x), \qquad  x \in \Bbb R \endcases $$

Use the D' Alembert formula to find the solution $u=u(x, t)$ in the following cases:
\smallskip

a) Speed $c=1$, $\phi(x)= \sin x$ and  $\psi(x)= 0$.  Calculate then $u_t(0, t)$.

\smallskip

b)  Speed $c= 2$, $\phi =0$, and  $\psi(x)= \cos x$

\smallskip

c)  Speed $c=1$, $\phi(x)=0$, and $$\psi(x) = \cases  1,  \qquad |x| < 3 \\ 0,  \qquad |x| \geq 3 \endcases $$ \underbar{Hint} Show 
$$ \int_{x-t}^{x+t} \psi(y) \, dy \,=\,  \text{length}[\,( x-t, x+t) \cap (-3, 3) \,].$$



\vskip .2in
$\bullet$\, {\bf{Set 8. Due date: Tuesday May 2nd}}
\vskip .2in

\underbar{Problem 1:}

\vskip .1in 
Let $u\,= \,u(x, t)$ be a solution to the wave equation $u_{tt} -  u_{xx} =0$ in $\Bbb R$.  Assuming that $ u_x \to 0$ fast enough as $| x| \to \infty$ prove that $$ E(t) \, =\, {\int_{\Bbb R} }\, \,  |u_t|^2 \, + \,  |u_x |^2 \, \, dx $$  is constant in $t$ for all time $t$. 
\vskip .2in

\underbar{Problem 2:}  Use Problem 1 together with the energy method to prove that solutions to the IVP for wave equation on $\Bbb R$
$$\qquad \cases u_{tt} \,- \,  c^2 u_{xx} \, =\, 0,   \qquad  x \in \Bbb R,   t>0  \\
  u(x, 0) \, =\, \phi(x), \quad   u_t(x, 0) = \psi (x), \qquad  x \in \Bbb R \endcases $$
are unique. 

\vskip .2in
\underbar{Problem 3:}

Find the solution to the wave equation $ u_{tt} - 4 u_{xx} =0$ \,  with initial conditions $u(x, 0) = \cos x, \, u_t(x, 0)= 0$.  Compute  $u_t(x,t)$ and then calculate then $u_t(0, t)$. 


\vskip .1in
\underbar{Problem 4:}  The following equation is a linear homogeneous wave equation on $\Bbb R$ in disguise. Find the general solution by following the indicated 
steps.
$$(\dagger) \qquad   3 u_{tt} - 4  u_{xt}  + u_{xx} \, =\, 0  $$ 
%  u(x, 0) \, =\, \phi(x), \quad   u_t(x, 0) = 0, \qquad  x \in \Bbb R \endcases $$
  
 i)  Consider the linear operator associated to the equation $$\Cal L =3 \partial^2_{t}  - 4 \partial_{t}  \partial_{x} +  \partial^2_{x} $$ and complete the square in terms of $\partial_x$ and $\partial_t$ to show that for suitable $a$ and $b$ ( which you need to explicitly find)   
$$\Cal L =  ( a \partial_x  - b  \partial_t)^2\, - \, \partial_{tt} = 0 $$ 
\medskip

ii) Next consider the change of variables   $$ z = x \qquad \qquad y = b x + a t $$ where $a$ and $b$ are those found in i) and prove that 
the equation $(\dagger)$ in the $z, y$ variables becomes 

$$(\dagger \dagger) \qquad   u_{zz} -  u_{yy} \, =\, 0. $$ 

\underbar{Hint}. Use chain rule to compute  $\partial_x$ in terms of $\partial_z$ and $\partial_y$,  and $\partial_t$ in terms of $\partial_z$ and $\partial_y$.
 
 \medskip
 
iii)  Write the general solution $u(z, y)$ to the equation $(\dagger \dagger)$  in the $z, y$ variables. Then undo the change of variables to 
find the solution $u(x,t)$ to the equation $(\dagger)$. 

\enddocument
\newpage 

\centerline{\bf To do (but do not turn in yet).} 
\bigskip

{\underbar{Final Problem 1}:  Find the solution $u(x,t)$ to the following inhomogeneous diffusion boundary/initial value problem  with Dirichlet boundary conditions (proceed as in handout example). 

\vskip .05in
  $$\cases u_{t} \,- \,  u_{xx} \, =\, e^{-t},   \qquad  0< x < \pi, \, \, \, t>0  \\
  u(x, 0) \, =\, 1, \quad 0< x<\pi \\
u(0, t) \,= \, 0 \quad \text{and} \quad u(\pi, t) \,=\, 0 \endcases $$

\underbar{Hints}  First find the sine Fourier series for $e^{-t}$ on $(0, \pi)$. Note that you had to find the sine Fourier series for $1$ in Set  2 Problem 4.  

At some point you'll encounter an ODE of the form  $a_n^{\prime}(t)  + c_1 n^2  a_n(t)  = c_2 \frac{e^{-t}}{n}$,  for some specific constants $c_1, c_2$. To find the solution to this ODE, consider $a_n(t) = A_n e^{-t}$ for suitable $A_n$ that you would need to find.

\bigskip


{\underbar{Final Problem 2}:  Prove the uniqueness of solutions to the Poisson equation on the whole space $\Bbb R^d$. 
Treat the case $d\geq 3$ first when solutions $u(x) \to 0$ as $|x| \to \infty$. Then treat separately the case $d=2$ where  uniqueness is under the assumptions $\frac{u(x)}{|x|} \to 0$ as $|x| \to \infty$ and $|\nabla u(x)| \to 0$ as $|x| \to \infty$.

\enddocument 


%{\underbar {Some more problems, related to the Laplace's equation will be added soon --this week.}} 

%{\underbar{Please start working on the above.}}


%\end{document}

$\bullet$\, {\bf{Set 7. Due date:  Tuesday May 6  }}
\vskip .1in 
\underbar{Section 5.2} \, \, \,  3, 4, 7, 9, 10, 11, 17.

\vskip .1in 
\underbar{Section 5.3} \, \, \, 1, 2a)b), 3, 4a)b), 5a), 6, 10.

\vskip .1in  

\underbar{Section 5.4} \, \, \,  1, 5, 6, 7 , 8a)



\vskip .2in 

$\bullet$\, {\bf{Set 6. Due date:  Thursday April 24 }}
\vskip .1in 
\underbar{Section 4.3} \, \, \,   2, 4, 6, 11

\vskip .1in 
\underbar{Section 5.1} \, \, \, 3, 5, 6, 7, 9. 
\vskip .1in 

\underbar{Additional Problems for 4.3 and 5.1} 

\vskip .2in 

$\bullet$\, {\bf{Set 5. Due date:  Thursday April 10 }}

\vskip .1in

\underbar{Section 2.4}:\quad  1, 2, 7 (here you can use that the integral is $\sqrt{\pi}$ which we proved in class) 

\hskip 0.8in 8, 14, 15 16, 18

\vskip .1in 

\underbar{Section 2.5}:\quad 1  

\vskip .1in 

\underbar{Section 4.1}:\quad  2, 3
\vskip .1in 

\underbar{Section 4.2}:\quad  1, 3

\vskip .1in




\underbar{Additional Problems} Separate variables to solve the following problems. 
\vskip .1in 

A1.  \quad  $u_{tt} - u_{xx} =0$ in $0<x<3$ with boundary conditions $u(0, t)= u(3, t)=0$

\vskip .1in
A2. \quad $u_t=u_{xx}$ in $0<x< L$ with boundary conditions $u_x(0,t)= u (L,t)=0$

\vskip .1in

A3. \quad \, $u_t \,=\, u_{xx}\,+\, u$ in $0< x< \pi$ with boundary conditions $u(0,t)\, = \,u(\pi, t) \,=\,0$
\vskip .2in 

$\bullet$\, {\bf{Set 4. Due date:  Thursday March 13th }}

\vskip .1in


\underbar{Section 2.3}:\quad 2, 4, 5, 6, 7. 

\vskip .2in 
$\bullet$\, {\bf{Set 3. Due date:  Thursday March 6th.\, \underbar{Postponed to March 11th}}}

\vskip .1in

\underbar{Section 2.1}:\quad 1, 2, 5,  8, 9, 11  

\vskip .1in

{\bf Problem 11.}   Find the general solution of $$ 3u_{tt} +10 u_{xt} + 3 u_{xx} \, =\, \sin( x+t ) $$

\vskip .1in 

\underbar{Section 2.2}:  1, 2, 3, 5* 
\vskip .1in
{\bf Problem 5*}  \, Consider the wave equation in $1$D with {\it damping} 
$$ u_{tt} = c^2\, u_{xx} - k u - r u_{t} \, \qquad k, r >0$$  show that the {\it energy functional} $$E(t) \, =\, \frac{1}{2} \int_{-\infty}^{\infty} \, |u_t|^2  + c^2 |u_x|^2 + k |u|^2 \, dx $$ satisfies $dE/dt \le 0$; that is {\it energy decreases}. Assume $u$ and its derivatives vanish as  $x \to \pm \infty$. 

\vskip .1in

{\bf Additional Problems}.
\vskip .1in 
{\bf Problem A.1.} \,\, Find the solution to wave equation with initial conditions $u(x, 0) = \sin x, \, u_t(x, 0)= 0$. Calculate then $u_t(0, t)$. 

\vskip .1in

{\bf Problem A.2.}\, \, Solve the PDE \quad $ u_{xx} + u_{xt} -  10 u_{tt} \,=\, 0$



\vskip .2in


$\bullet$\, {\bf{Set 2. Due date:  Thursday February 28th }}
\vskip .2in

\underbar{Section 1.4}:  \quad  1

\vskip .1in 

\underbar{Section 1.5}:  \quad 1,  4,  5,  6 (modified; see below)

\vskip .2in 

\underbar{Problem 5 states}:  Consider the equation $$u_x + y u_y =0$$ 

with boundary condition $u(x,0)= \phi(x)$.

(a) \, For $\phi(x)= x$, for all $x$, show that no solution exists.

(b) \, For $\phi(x)=1$ for all $x$, show that there are many solutions. 

\vskip .1in

\underbar{Problem 6 states}:  Solve the equation $u_x \, +\, 2 x \, y^2\, u_y \,=\,0$ and find a solution that satisfies the auxiliary condition $u(0, y) =  y$.  

\vskip .2in 

\underbar{Section 1.6}:  \quad 1, 2, 4

\vskip .2in 
\underbar{Additional Problems:} 

(1) \, Find the general solution of $u_x - \sin(x)\, u_y \,=\,0$. 
Then find the special solution that satisfies $f(0, e^y)= e^{y}$.   

\vskip .1in

(2) \, Find the regions in 2d-space where $x^2\, u_{xx} + 4 u_{xy} +  y^2 \, u_{yy} \, =\,0$ 
is respectively elliptic, parabolic, hyperbolic. Plot these regions. 



\vskip .2in


$\bullet$\, {\bf{Set 1. Due date:  Thursday February 14th }}
\vskip .2in


\underbar{Section 1.1}:  \quad 2, 3, 4, 11

\vskip .1in 

\underbar{Section 1.2}:  \quad   2, 3, 4, 5.

\vskip .1in 

\underbar{Section 1.3}: \quad 7,  9,  10 (in $\Bbb R^3$), 11.

\vskip .2in 

\underbar{Additional Problem:}

\vskip .1in 
Let $u\,= \,u(\bold{x}, t)$ be a solution to the wave equation $u_{tt} - \Delta u =0$ in $\Bbb R^2$.  Assuming that $ \nabla u \to 0$ fast enough as $| \bold{x}| \to \infty$ prove that $$ E(t) \, =\, {\int}_{\Bbb R^2} \, \,  |u_t|^2 \, + \,  |\nabla\, u |^2 \, \, dx dy $$  is constant in $t$ for all time $t$. 

\enddocument





                                               

 \newpage 
 \ 
${\bold 2}$)  Consider the solution to the initial value problem for the wave equation on $\IR$:
$$ \cases u_{tt} - u_{xx} \,=\, 0  \qquad x \in \IR, \\
u(x, 0) = 0 \\
u_t (x, 0) = \chi_{[-2, 2]}(x)  \quad \text{this is }  1 \text{ on }  |x| \le 2; \text{ and } \, 0 \, \text{ otherwise. } \endcases $$ 


(a) Use D' Alembert's formula to write down the solution $u(x,t)$ in terms of its initial data (do not evaluate the integral).
\vskip 1in

(b) Use basic rules for integrals to compute $u_t (x,  t)$. 
\vskip 3.5in
(c)  Set $x=0$ in (b) and prove that for all $|t|>2$ we have that $u_t (0,  t) \,=\, 0$. 

\newpage














                                               













